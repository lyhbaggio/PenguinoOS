@comment -*-texinfo-*-
@comment This file was generated by doc2tex.pl from general.doc
@comment DO NOT EDIT DIRECTLY, BUT EDIT general.doc INSTEAD
@comment Id: general.tex,v 1.1 2003/08/08 14:27:06 pertusus Exp $
@comment this file contains the general information about Singular

@c The following directives are necessary for proper compilation
@c with emacs (C-c C-e C-r).  Please keep it as it is.  Since it
@c is wrapped in `@ignore' and `@end ignore' it does not harm `tex' or
@c `makeinfo' but is a great help in editing this file (emacs
@c ignores the `@ignore').
@ignore
%**start
\input texinfo.tex
@setfilename general.hlp
@node Top, General concepts
@menu
* General concepts::
@end menu
@node General concepts, Data types, Introduction, Top
@chapter General concepts
%**end
@end ignore

@menu
* Interactive use::
* Emacs user interface::
* Rings and orderings::
* Implemented algorithms::
* The SINGULAR language::
* Input and output::
* Procedures::
* Libraries::
* Guidelines for writing a library::
* Debugging tools::
@end menu


@c ---------------------------------------------------------------------------
@node Interactive use, Emacs user interface, General concepts, General concepts
@section Interactive use
@cindex Interactive use

In this section, aspects of interactive use are discussed.  This
includes how to enter and exit @sc{Singular}, how to interpret its
prompt, how to get the online help, and so on.

There are a few important notes which one should not forget:
@itemize @bullet
@item
every command has to be terminated by a @code{;} (semicolon) followed
by a @key{RETURN}
@item
the online help is accessible by means of the @code{help} function
@end itemize

@menu
* How to enter and exit::
* The SINGULAR prompt::
* The online help system::
* Interrupting SINGULAR::
* Editing input::
* Command line options::
* Startup sequence::
@end menu

@c --------------------------------------------------------------------------
@node How to enter and exit, The SINGULAR prompt, Interactive use, Interactive use
@subsection How to enter and exit
@cindex How to enter and exit

@sc{Singular} can either be run in ASCII-terminal or within Emacs.

To start @sc{Singular} in its ASCII-terminal user interface,  enter
@code{Singular} at the system prompt. The
@sc{Singular} banner appears which, among others, reports the version
and the compilation date.

To start @sc{Singular} in its Emacs user interface, either enter
@code{ESingular} at the system prompt, or type @code{M-x singular}
within a running Emacs (provided you have loaded the file
@code{singular.el} in your running Emacs, see @ref{Running SINGULAR
under Emacs} for details).

Generally, we recommend to use @sc{Singular} in its Emacs interface,
since this offers many more features and is more convenient to
use than the ASCII-terminal interface (@pxref{Emacs user interface}).

To exit @sc{Singular} type @code{quit;}, @code{exit;} or @code{$} (or,
when running within Emacs preferably type @code{C-c $}).

@sc{Singular} and @code{ESingular} may also be started with command line
options and with
filenames as arguments. More generally, the synopsis of @sc{Singular} is
@smallexample
Singular  [options] [file1 [file2 @dots{}]]
ESingular  [options] [file1 [file2 @dots{}]]
@end smallexample

@xref{Command line options}, @ref{Startup sequence}, @ref{Running SINGULAR
under Emacs}.


@c --------------------------------------------------------------------------
@node The SINGULAR prompt,The online help system,How to enter and exit, Interactive use
@subsection The SINGULAR prompt
@cindex prompt
@cindex error recovery
The @sc{Singular} prompt @code{>} (larger than) asks the user for input
of commands.  The ``continuation'' prompt @code{.} (period) asks the
user for input of missing parts of a command (remember the semicolon at
the end of every command).

@sc{Singular} does not interpret the semicolon as the end of a command
if it occurs inside a string.  Also, @sc{Singular} waits for blocks
(sequences of commands enclosed in curly brackets) to be closed before
prompting with @code{>} for more commands.  Thus, if @sc{Singular} does
not respond with its regular prompt after typing a semicolon it may wait
for a @code{"} or a @code{@}} first.

Additional semicolons will not harm @sc{Singular} since they are
interpreted as empty statements.

@c --------------------------------------------------------------------------
@node The online help system,Interrupting SINGULAR, The SINGULAR prompt,Interactive use
@subsection The online help system
@cindex The online help system
@cindex online help
@cindex help, online help system

The online help system is invoked with the @code{help} command.
@code{?} may be used as a synonym for @code{help}.  Simply typing
@code{help;} displays the ``top'' of the help system (i.e., the title
page of the @sc{Singular} manual) which offers a
short table of contents.  Typing @code{help} topic@code{;} shows the
available documentation on topic.  Here, topic may be either a function
name or, more generally, any index entry of the @sc{Singular}
manual. Furthermore, topic may contain wildcard characters.
@xref{help}, for more information.

@cindex browsers
@cindex help browsers
Online help information can be displayed in various help browsers. The
following table lists a summary of the supported browsers. See below for
more details about particular browsers.

@multitable @columnfractions .2 .2 .6
@item @strong{Browser} @tab @strong{Platform}
@tab @strong{Description}

@item html
@cindex help browsers, html
@cindex html, default help
@tab Windows
@tab displays html version of manual in your default html browser

@item winhlp
@cindex help browsers, winhlp
@cindex html, default help
@tab Windows
@tab displays Windows help (.hlp) version of manual

@item netscape
@cindex netscape
@cindex help browsers, netscape
@tab Unix, Windows
@tab uses @code{netscape} to display html version of manual

@item tkinfo
@cindex tkinfo
@cindex help browsers, tkinfo
@tab Unix
@tab uses the @code{tkinfo} program to display texinfo version of manual

@item xinfo
@cindex xinfo
@cindex help browsers, xinfo
@tab Unix, Windows
@tab uses the @code{info} program to display texinfo version of manual in an
extra @code{xterm}

@item info
@cindex info
@cindex help browsers, info
@tab Unix, Windows
@tab uses the @code{info} program to display texinfo version of manual

@item builtin
@cindex help browsers, dummy
@tab all
@tab simply outputs the help information in plain ASCII format

@item emacs
@cindex help browsers, emacs
@tab Unix, Windows
@tab when running @sc{Singular} within (X)emacs, display help in
(X)emacs info buffer.

@item dummy
@tab all
@cindex help browsers, dummy
@tab displays error message about no available help browser
@end multitable

The browser which is used to display the help information, can be either
set at startup time with the command line option  (@pxref{Command line
options})
@smallexample
--brower=<browser>
@end smallexample
or with the @sc{Singular} command (@pxref{system})
@smallexample
system("--browser", <browser>);
@end smallexample
The  @sc{Singular} command
@smallexample
system("browsers");
@end smallexample
lists all available browsers and the command
@smallexample
system("--browser");
@end smallexample
returns the currently used browser.

If no browser is explicitly set by the user, then the first (w.r.t. the
order of the browsers in the table above) available browser is
chosen. In other words, the default browser is @code{netscape} for Unix
platforms, and @code{html} for Windows, unless
your local set-up does not allow to run it.

The @code{.singularrc} (@pxref{Startup sequence}) file is a good place
to set your default browser. Recall that if a file
@code{$HOME/.singularrc} exists on your system, then the content of this
file is executed before the first user input. Hence, putting
@smallexample
if (system("version") >= 1306) // keep backwards-compatibility
@{
  if (! system("--emacs")
  @{
    // only set help browser if not running within emacs
    system("--browser", "info");
  @}
  // if help browser is later on set to netscape,
  // allow it to fetch HTML pages from the net
  system("--allow-net", 1);
@}; // the last semicolon is important: otherwise no ">", but "." prompt
@end smallexample
in your file @code{$HOME/.singularrc} sets your default browser to
@code{info}, unless @sc{Singular} is run within emacs (in which case the
default browser is set automatically to @code{emacs}).


Obviously, certain external files and programs are required for the
@sc{Singular} help system to work correctly. If something is not available
or goes wrong, here are some tips for troubleshooting the help system:

@itemize @bullet
@item
@cindex DISPLAY environment variable
@cindex environment variable, DISPLAY
Under Unix, the environment variable @code{DISPLAY} has to be set for
the browsers @code{netscape}, @code{tkinfo}, and @code{xinfo} to work.

@item
@cindex NETSCAPE environment variable
@cindex XINFO environment variable
@cindex XTERM environment variable
@cindex INFO environment variable
@cindex help browsers, setting command to use
The help browsers @code{netscape}, @code{tkinfo}, @code{xinfo}, and
@code{info} are only available if the respective programs are installed
on your system (for @code{xinfo}, the programs @code{xterm} and
@code{info} are necessary). You can explicitly specify which program to
use, by setting appropriate environment variables (whose names
are the names of the respective programs in all up-case letters).
@*For example,
to use the program @code{navigator} with a private colormap as
@code{netscape} help browser, set the environment variable
@code{NETSCAPE} to the value
@code{"navigator -install"}.
@c TBC (@xref{Setting environment variables}).

@item
If the help browser @code{netscape} can not find the local html
pages of the @sc{Singular} manual (which it looks for at
@code{$RootDir/html} -- see @ref{Loading of a library} for more info on
@code{$RootDir}) @emph{and} the (command-line) option @code{--allow-net}
has @emph{explicitly} been set (see @ref{Command line options} and
@ref{system} for more info on
setting values of command-line options), then it dispatches the html
pages from
@url{http://www.singular.uni-kl.de/Manual} (note that
the non-local net-access of HTML pages is disabled, by default).
@*An
alternative location of a local directory where the html pages reside
can be specified by setting the environment variable
@code{SINGULAR_HTML_DIR}.
@c TBC (@xref{Setting environment variables}).

@item
The help browsers @code{tkinfo}, @code{xinfo}, @code{info}, and
@code{builtin} need the (info) file @code{singular.hlp} which it looks
for at @code{$RootDir/info/singular.hlp} (see @ref{Loading of a library}
for more info on @code{$RootDir}).  An alternative
location of the info file of the manual can be specified by setting the
environment variable @code{SINGULAR_INFO_FILE}.
@c TBC (@xref{Setting environment variables}).
@end itemize

@c inserted refs from general.doc:308
@ifinfo
@menu
See also:
* Command line options::
* Loading of a library::
* Startup sequence::
* system::
@end menu
@end ifinfo
@iftex
@strong{See also:}
@ref{Command line options};
@ref{Loading of a library};
@ref{Startup sequence};
@ref{system}.
@end iftex
@c end inserted refs from general.doc:308

@subsubheading Info help browsers
@cindex info

The help browsers @code{tkinfo}, @code{xinfo} and @code{info} (so-called
info help browsers) are based on the
@code{info} program from the Gnu @code{texinfo} package.  @xref{Top, An
Introduction to Info, Getting started, info, The Info Manual}, for more
information.

For info help browsers, the online manual is decomposed into ``nodes''
of information, closely
related to the division of the printed manual into sections and
subsections.  A node contains text describing a specific topic at a
specific level of detail.  The top line of a node is its ``header''.
The node's header tells the name of the current node (@code{Node:}), the
name of the next node (@code{Next:}), the name of the previous node
(@code{Prev:}), and the name of the upper node (@code{Up:}).

To move within info, type commands consisting of single characters.  Do
not type @code{RETURN}.  Do not use cursor keys, either.  Using some of
the cursor keys by accident might pop to some totally different node.
Type @code{l} to return to the original node.  Some of the @code{info}
commands read input from the command line at the bottom line.  The
@code{TAB} key may be used to complete partially entered input.

The most important commands are:
@table @asis
@item @code{q}
leaves the online help system
@item @code{n}
goes to the next node
@item @code{p}
goes to the previous node
@item @code{u}
goes to the upper node
@item @code{m}
picks a menu item specified by name
@item @code{f}
follows a cross reference
@item @code{l}
goes to the previously visited node
@item @code{b}
goes to the beginning of the current node
@item @code{e}
goes to the end of the current node
@item @code{SPACE}
scrolls forward a page
@item @code{DEL}
scrolls backward a page
@item @code{h}
invokes info tutorial (use @code{l} to return to the manual or
@code{CTRL-X 0} to remove extra window)
@item @code{CTRL-H}
shows a short overview on the online help system (use @code{l} to return
to the manual or @code{CTRL-X 0} to remove extra window)
@item @code{s}
searches through the manual for a specified string, and selects the node in
which the next occurrence is found
@item @code{1}, @dots{}, @code{9}
picks i-th subtopic from a menu
@end table

@c --------------------------------------------------------------------------
@node Interrupting SINGULAR,Editing input,The online help system,Interactive use
@subsection Interrupting SINGULAR
@cindex Interrupting SINGULAR

On Unix-like operating systems and on Windows NT, typing @code{CTRL-C}
(or, alternatively @code{C-c C-c}, when running within Emacs),
interrupts @sc{Singular}.  @sc{Singular} prints the current command and
the current line and prompts for further action.  The following choices
are available:
@table @code
@item a
returns to the top level after finishing the current (kernel) command. Notice
that commands of the @sc{Singular} kernel (like @code{std}) can not be
aborted, i.e. (@code{a})bort only happens whenever the interpreter is active.
@item c
continues
@item q
quits  @sc{Singular}
@end table

@c --------------------------------------------------------------------------
@node Editing input,Command line options,Interrupting SINGULAR,Interactive use
@subsection Editing input
@cindex Editing input
@cindex SINGULARHIST

@c This section describes only a subset of the key bindings of
@c @sc{Singular} binaries built with the GNU Readline library.
@c @xref{Command Line Editing, GNU Readline Library, Command Line Editing,
@c readline, The GNU Readline Library Manual}, for more information.

The following keys can be used for editing the input and retrieving
previous input lines:
@table @code
@item TAB
provides command line completion for function names and file names
@item CTRL-B
moves cursor left
@item CTRL-F
moves cursor right
@item CTRL-A
moves cursor to beginning of line
@item CTRL-E
moves cursor to end of line
@item CTRL-D
deletes character under cursor
@* Warning: on an empty line, @code{CTRL-D} is interpreted as the
@code{EOF} character which immediately terminates @sc{Singular}.
@item BACKSPACE
@itemx DELETE
@itemx CTRL-H
deletes character before cursor
@item CTRL-K
kills from cursor to end of line
@item CTRL-U
kills from cursor to beginning of line
@item CTRL-N
saves current line on history and gives next line
@item CTRL-P
saves current line on history and gives previous line
@item RETURN
saves the current line to the history and
sends it to the @sc{Singular} parser for interpretation
@end table

When run under a Unix-like operating system and in its ASCII-terminal
user interface,   @sc{Singular} tries to dynamically link at runtime
with the GNU Readline library. @xref{Command Line Editing, GNU Readline
Library, Command Line Editing, readline, The GNU Readline Library
Manual}, for more information. If a shared version of this library can
be found on your machine, then additional command-line editing
features like  history completion are available.
In particular, if @sc{Singular} is able to load that library and
if the environment
variable @code{SINGULARHIST} is set and has a name of a file as value,
then the input history is stored across sessions using this
file. Otherwise, i.e., if the environment variable @code{SINGULARHIST}
is not set, then the history of the last inputs is only available for
the commands of the current session.

@c --------------------------------------------------------------------------
@node Command line options, Startup sequence, Editing input, Interactive use
@subsection Command line options
@cindex Command line options

The synopsis of @sc{Singular} is
@smallexample
Singular  [options] [file1 [file2 @dots{}]]
ESingular  [options] [file1 [file2 @dots{}]]
@end smallexample
Options can be given in both, their long and short format. The following
options control the general behavior of @sc{Singular}:
@table @asis

@item @code{-d}, @code{--sdb}
@cindex -d
@cindex --sdb
@cindex Source code debugger, invocation
Enable the use of the source code debugger.
@xref{Source code debugger}.

@item @code{-e}, @code{--echo[=VAL]}
@cindex -e
@cindex --echo
Set value of variable @code{echo} to @code{VAL} (integer in the range
0, @dots{}, 9). Without an argument, @code{echo} is set to 1, which echoes
all input coming from a file. By default, the value of @code{echo} is
0. @xref{echo}.

@item @code{-h}, @code{--help}
@cindex -h
@cindex --help
@cindex command-line options, short help
Print a one-line description of each command line option and exit.

@item @code{--allow-net}
@cindex --allow-net
@cindex net access
@cindex allowing net access
@cindex help, accessing over the net
Allow the @code{netscape} and @code{html} help browser to fetch HTML manual pages over
the net from the WWW home-site of @sc{Singular}. @xref{The online help
system}, for more info.

@item @code{--browser=VAL}
@cindex --browser
@cindex browser, command line option
Use @code{VAL} as browser for the @sc{Singular} online manual.
@*@code{VAL} may be one of @code{html} (Windows only), @code{netscape}, @code{xinfo},
@code{tkinfo}, @code{info}, @code{builtin}, or @code{emacs}. Depending
on your platform and local
installation, only some browsers might be available. The default browser
is @code{html} for Windows and @code{netscape} for Unix
platforms. @xref{The online help system}, for more info.

@item @code{--no-rc}
@cindex --no-rc
@cindex .singularrc file, no loading
Do not execute the @code{.singularrc} file on start-up.  By default,
this file is executed on start-up.  @xref{Startup sequence}.

@item @code{--no-stdlib}
@cindex --no-stdlib
Do not load the library @code{standard.lib} on start-up. By default,
this library is loaded on start-up.  @xref{Startup sequence}.

@item @code{--no-warn}
@cindex --no-warn
Do not display warning messages.

@item @code{--no-out}
@cindex --no-out
Suppress display of all output.

@item @code{-t}, @code{--no-tty}
@cindex --no-tty
Do not redefine the terminal characteristics.  This option should be
used for batch processes.

@item @code{-q}, @code{--quiet}
@cindex -q
@cindex --quiet
Do not print the start-up banner and messages when loading
libraries. Furthermore, redirect @code{stderr}
(all error messages) to @code{stdout} (normal output channel).  This
option should be used if @sc{Singular}'s output is redirected to a file.

@item @code{-v}, @code{--verbose}
@cindex --verbose
@cindex -v
Print extended information about the version and configuration of
@sc{Singular}  (used optional parts, compilation date, start of random
generator etc.). This information should be included if a user reports
an error to the authors.
@end table

The following command line options allow manipulations of the timer and
the pseudo random generator and enable the passing of commands and strings
to @sc{Singular}:

@table @asis
@item @code{-c}, @code{--execute=STRING}
@cindex -c
@cindex --execute
Execute @code{STRING} as (a sequence of) @sc{Singular} commands on
start-up after the @code{.singularrc} file is executed, but prior to
executing the files given on the command line.  E.g., @code{Singular -c
"help all.lib; quit;"} shows the help for the library @code{all.lib} and
exits.

@item @code{-u}, @code{--user-option=STRING}
@cindex -u
@cindex --user-option
Returns @code{STRING} on @code{system("--user-option")}. This is useful
for passing arbitrary arguments from the command line to the
@sc{Singular} interpreter.  E.g.,
@*@code{Singular -u "xxx.dump" -c 'getdump(system("--user-option"))'}
reads the file @code{xxx.dump} at
start-up and allows the user to start working with all the objects
defined in a previous session.

@item @code{-r}, @code{--random=SEED}
@cindex -r
@cindex --random
Seed (i.e., set the initial value of) the pseudo random generator with
integer @code{SEED}.  If this option is not given, then the random
generator is seeded with a time-based @code{SEED} (the number of
seconds since January, 1, 1970, on Unix-like operating systems, to be
precise).

@item @code{--min-time=SECS}
@cindex --min-time
If the @code{timer} (@pxref{timer}), resp.@:  @code{rtimer}
(@pxref{rtimer}) , variable is
set, report only
times larger than @code{SECS} seconds (@code{SECS} needs to be a
floating point number greater than 0).  By default, this value is set to
0.5 (i.e., half a second).  E.g., the option @code{--min-time=0.01}
forces @sc{Singular} to report all times larger than 1/100 of a
second.

@item @code{--ticks-per-sec=TICKS}
@cindex --ticks-per-sec
Set unit of timer to @code{TICKS} ticks per second (i.e., the value
reported by the @code{timer} and @code{rtimer} variable divided by
@code{TICKS} gives the time in seconds).  By default, this value is 1.

@end table

The next three options are of interest for the use with MP links:
@table @asis
@item @code{-b}, @code{--batch}
@cindex -b
@cindex --batch
Run in MP batch mode. Opens a TCP/IP connection with host specified by
@code{--MPhost} at the port specified by @code{--MPport}. Input is read
from and output is written to this connection in the MP format.
@xref{MPtcp links}.

@item @code{--MPport=PORT}
@cindex --MPport
Use @code{PORT} as default port number for MP connections (whenever not
further specified). This option is mandatory when the @code{--batch}
option is given.  @xref{MPtcp links}.

@item @code{--MPhost=HOST}
@cindex --MPhost
Use @code{HOST} as default host for MP connections (whenever not
further specified).  This option is mandatory when the @code{--batch}
option is given.  @xref{MPtcp links}.
@end table

Finally, the following options are only available when running
@code{ESingular} (see @ref{Running SINGULAR under Emacs} for details).

@table @code

@item --emacs=EMACS
@cindex --emacs
Use @code{EMACS} as Emacs program to run the @sc{Singular} Emacs
interface, where @code{EMACS} may e.g. be emacs ore xemacs.

@item --emacs-dir=DIR
@cindex --emacs-dir
Set the singular-emacs-home-directory, which is the directory where
singular.el can be found, to @code{DIR}.

@item --emacs-load=FILE
@cindex --emacs-load
Load @code{FILE} on Emacs start-up, instead of the default load file.

@item --singular=PROG
@cindex --singular
Start @code{PROG} as @sc{Singular} program within Emacs
@end table

The value of options given to @sc{Singular} (resp.@: their default values,
if an option was not given), can be checked with the command
@code{system("--}long_option_name@code{")}.  @xref{system}.
@smallexample
@c computed example Command_line_options general.doc:658 
  system("--quiet");    // if ``quiet'' 1, otherwise 0
@expansion{} 1
  system("--min-time"); // minimal reported time
@expansion{} 0.5
  system("--random");   // seed of the random generator
@expansion{} 12345678
@c end example Command_line_options general.doc:658
@end smallexample

Furthermore, the value of options (e.g., @code{--browser}) can be
re-set while @sc{Singular} is running with the command
@code{system("--}long_option_name_string @code{",}expression@code{)}. @xref{system}.
@smallexample
  system("--browser", "builtin");  // sets browser to 'builtin'
  system("--ticks-per-sec", 100);  // sets timer resolution to 100
@end smallexample


@c --------------------------------------------------------------------------

@node Startup sequence, , Command line options, Interactive use
@subsection Startup sequence
@cindex Startup sequence

On start-up, @sc{Singular}
@enumerate
@item
loads the library @code{standard.lib} (provided the @code{--no-stdlib}
option was not given),

@item
@cindex file, .singularrc
@cindex .singularrc file
@cindex singularrc
searches the current directory and then the
home directory of the user, and then all directories contained in the
library @code{SearchPath} (see @ref{Loading of a library} for more info
on @code{SearchPath}) for a file named @code{.singularrc} and
executes it, if found (provided the @code{--no-rc} option was not
given),

@item
executes the string specified with the @code{--execute} command line
option,

@item
executes the files @code{file1}, @code{file2} @dots{}  (given on the command
line) in that order.
@end enumerate

@strong{Note:} @code{.singularrc} file(s) are an appropriate place for
setting some default values of (command-line) options.

For example, a system administrator might remove the locally installed
HTML version of the manual and put a @code{.singularrc} file
with the following content
@smallexample
if (system("version") >= 1306) // assure backwards-compatibility
@{
  system("--allow-net", 1);
@}; // the last semicolon is important: otherwise no ">", but "." prompt
@end smallexample
in the directory containing the @sc{Singular} libraries, thereby
allowing to fetch the HTML on-line help from the WWW home-site of
@sc{Singular}.

On the other hand, a single user might put a @code{.singularrc} with the
following content
@smallexample
if (system("version") >= 1306) // assure backwards-compatibility
@{
  if (! system("--emacs"))
  @{
    // set default browser to info, unless we run within emacs
    system("--browser", "info");
  @}
@}; // the last semicolon is important: otherwise no ">", but "." prompt
@end smallexample
in his home directory, which sets the default help browser to
@code{info} (unless @sc{Singular} is run within emacs) and thereby
prevents the execution of the"global"
@code{.singularrc} file installed by the system administrator (since the
@code{.singularrc} file of the user is found before the "global"
@code{.singularrc} file installed by the system administrator).

@c ---------------------------------------------------------------------------
@node Emacs user interface, Rings and orderings, Interactive use, General concepts
@section Emacs user interface
@cindex Emacs, user interface
@cindex Emacs
@cindex interface, Emacs
@cindex user interface, Emacs

Besides running @sc{Singular} in an ASCII-terminal, @sc{Singular} might
also be run within Emacs. Emacs (or, XEmacs which is very similar) is a
powerful and freely available text editor, which, among others,
provides a framework for the implementation of interactive user
interfaces. Starting from version 1.3.6, @sc{Singular} provides such an
implementation, the so-called @sc{Singular} Emacs mode, or Emacs user
interface.

Generally, we recommend to use the Emacs interface,
instead of the ASCII-terminal interface: The Emacs interface does not
only provide everything the ASCII-terminal interface provides, but
offers much more. Among others, it offers
@itemize @bullet
@item
color-highlighting
@item
truncation of long lines
@item
folding of input and output
@item
TAB-completion for help topics
@item
highlighting of matching parentheses
@item
key-bindings and interactive menus for most user interface commands and
for basic @sc{Singular} commands (such as loading of libraries and files)
@item
a mode for running interactive @sc{Singular} demonstrations
@item
convenient ways to edit @sc{Singular} input files
@item
interactive customization of nearly all aspects of the user-interface.
@end itemize

To use the @sc{Singular}-Emacs interface you need to have Emacs version
20 or higher, or XEmacs
version 20.3 or higher installed on your system. These
editors can be downloaded for most hard- and software platforms
(including Windows
95/98/NT, but excluding the Macintosh), from either
@uref{http://www.fsf.org/software/emacs/emacs.html} (Emacs), from
@uref{http://www.xemacs.org} (XEmacs), or from our ftp site at
@uref{ftp://www.mathematik.uni-kl.de/pub/Math/Singular/utils/}. The
differences between Emacs and XEmacs w.r.t. the  @sc{Singular}-Emacs
interface are marginal -- which editor to use is mainly a matter of personal
preferences.

The simplest way to start-up @sc{Singular} in its Emacs interface is by
running the program @code{ESingular} which is contained in the Singular
distribution. Alternatively, @sc{Singular} can be started within an
already running Emacs -- see @ref{Running SINGULAR under Emacs} for
details.

The next section gives  a tutorial-like introduction to Emacs. This
introductory section is
followed by sections which explain the functionality of various aspects
of the Emacs user interface in more detail: how to start/restart/kill
@sc{Singular} within Emacs, how to run an interactive demonstration, how
to customize the Emacs user interface, etc. Finally, the 20 most
important commands of the Emacs interface together with their key
bindings are listed.

@menu
* A quick guide to Emacs::
* Running SINGULAR under Emacs::
* Demo mode::
* Customization of the Emacs interface::
* Editing SINGULAR input files with Emacs::
* Top 20 Emacs commands::
@end menu

@c --------------------------------------------------------------------------
@node A quick guide to Emacs, Running SINGULAR under Emacs, Emacs user interface, Emacs user interface
@subsection A quick guide to Emacs
@cindex Emacs, overview
@cindex Emacs, a quick guide

This section gives a tutorial-like introduction to Emacs. Especially to
users who are not familiar with Emacs, we recommend that they go through
this section and try out the described features.

Emacs commands generally involve the @code{CONTROL} key (sometimes
labeled @code{CTRL} or @code{CTL}) or the @code{META} key.  On some
keyboards, the @code{META} key is labeled @code{ALT} or @code{EDIT} or
something else (for example, on Sun keyboards, the diamond key to the
left of the space-bar is @code{META}).  If there is no @code{META} key,
the @code{ESC} key can be used, instead.  Rather than writing out
@code{META} or
@code{CONTROL} each time we want to prefix a character, we will use the
following abbreviations:

@multitable @columnfractions .2 .8
@item @code{C-<chr>}
@tab means hold the @code{CONTROL} key while typing the character
@code{<chr>}. Thus, @code{C-f} would be: hold the @code{CONTROL} key and
type @code{f}.
@item @code{M-<chr>}
@tab means hold the @code{META} key down while typing @code{<chr>}.  If there
is no @code{META} key, type @key{ESC}, release it, then type the
character @code{<chr>}.
@end multitable

For user new to Emacs, we highly recommend that they go through the
interactive Emacs tutorial: type @code{C-h t} to start it.

For others, it is important to understand the following Emacs concepts:
@table @asis
@item window
In Emacs terminology, a window refers to separate panes within the same
window of the window system, and not to overlapping, separate
windows. When using @sc{Singular} within Emacs, extra windows may appear
which display help or output from certain commands. The most important
window commands are:
@multitable @columnfractions 2. .3 .5
@item @code{C-x 1}
@tab @code{File->Un-Split}
@tab Un-Split window (i.e., kill other windows)
@item @code{C-x o}
@tab
@tab Goto other window, i.e. move cursor into other window.
@end multitable

@item cursor and point
The location of the cursor in the text is also called "point".  To
paraphrase, the cursor shows on the screen where point is located in
the text. Here is a summary of simple cursor-moving operations:
@multitable @columnfractions .2 .8
@item @code{C-f} @tab   Move forward a character
@item @code{C-b} @tab   Move backward a character
@item @code{M-f} @tab   Move forward a word
@item @code{M-b} @tab   Move backward a word
@item @code{C-a} @tab   Move to the beginning of line
@item @code{C-e} @tab   Move to the end of line
@end multitable

@item buffer
Any text you see in an Emacs window is always part of some buffer. For
example, each file you are editing with Emacs is stored inside a buffer,
but also @sc{Singular} is running inside an Emacs buffer. Each buffer
has a name: for example, the buffer of a file you edit usually has the
same name as the file, @sc{Singular} is running in a buffer which has
the name @code{*singular*} (or, @code{*singular<2>*},
@code{*singular<3>*}, etc., if you have multiple @sc{Singular}
sessions within the same Emacs).

When you are asked for input to an Emacs command, the cursor
moves to the bottom line of Emacs, i.e., to a special buffer, called the
"minibuffer". Typing @key{RETURN} within the minibuffer, ends the
input, typing @key{SPACE} within the minibuffer, lists all possible
input values to the interactive Emacs command.

The most important buffer commands are
@multitable @columnfractions .2 .8
@item @code{C-x b} @tab Switch buffer
@item @code{C-x k} @tab Kill current buffer
@end multitable
Alternatively, you can switch to or kill buffers using the @code{Buffer}
menu.

@item Executing commands
Emacs commands are executed by typing @code{M-x <command-name>}
(remember that @key{SPACE} completes partial
command names). Important and frequently used commands have short-cuts
for their execution: Key bindings or even menu entries. For example, a
file can be loaded with @code{M-x load-file}, or @code{C-x C-f}, or with
the @code{File->Open} menu.

@item How to exit
To end the Emacs (and, @sc{Singular}) session,
type @code{C-x C-c} (two characters), or use the @code{File -> Exit}
menu.

@item When Emacs is hung
If Emacs stops responding to your commands, you can stop it safely by
typing @code{C-g}, or, if this fails, by typing @code{C-]}.

@item More help
Nearly all aspects of Emacs are very well documented: type
@code{C-h}  and then a character saying what kind of help you want.
For example, typing @code{C-h i} enters the @code{Info} documentation
browser.

@item Using the mouse
Emacs is fully integrated with the mouse. In particular, clicking the
right mouse button brings up a pop-up menu which usually contains a few
commonly used commands.
@end table

@c --------------------------------------------------------------------------
@node Running SINGULAR under Emacs, Demo mode, A quick guide to Emacs, Emacs user interface
@subsection Running SINGULAR under Emacs
@cindex Emacs, running Singular under
@cindex Singular, running within Emacs
@cindex Running SINGULAR under Emacs

There are two ways to start the @sc{Singular} Emacs interface: Typing
@code{ESingular} instead of @code{Singular} on the command shell
launches a new Emacs process, initializes the interface and runs
@sc{Singular} within Emacs. The other way is to start the interface in
an already running Emacs, by typing @code{M-x singular} inside
Emacs. This initializes the interface and runs @sc{Singular} within
Emacs. Both ways are described in more detail below.

Note: To properly run the Emacs interface, several files are needed
which usually reside in the @code{emacs} subdirectory of your
@sc{Singular} distribution. This directory is called
singular-emacs-home-directory in the following.

@subheading Starting the interface using ESingular

As mentioned above, @code{ESingular} is an "out-of-the-box" solution:
You don't have to add special things to your @code{.emacs} startup file
to initialize the interface; everything is done for you in a special
file called @code{.emacs-singular} (which comes along with the
@sc{Singular} distribution and resides in the
singular-emacs-home-directory) which is automatically loaded on Emacs
startup (and the loading of the @code{.emacs} file is automatically
suppressed).

The customizable variables of the @sc{Singular} Emacs interface are set
to defaults which give the novice user a very shell like
feeling of the interface. Nevertheless, these default settings can be
changed, see @ref{Customization of the Emacs interface}. Besides other
Emacs initializations, such as fontification or blinking parentheses, a
new menu item called @code{Singular} is added to the main menu,
providing menu items for starting @sc{Singular}. On XEmacs, a button
starting @sc{Singular} is added to the main toolbar.

The @sc{Singular} interface is started automatically; once you see a
buffer called @code{*singular*} and the @sc{Singular} prompt you are
ready to start your @sc{Singular} session.

@code{ESingular} inherits all @code{Singular} options. For a description
off all these options, see @ref{Command line options}. Additionally
there are the following options which are special to @code{ESingular}:

@multitable @columnfractions .3 .7

@item command-line option /
@*environment variable
@tab functionality

@item @code{--emacs=EMACS}
@*@code{ESINGULAR_EMACS}
@tab Use @code{EMACS} as Emacs program to run the @sc{Singular} Emacs
interface, where @code{EMACS} may e.g. be emacs or xemacs.

@item @code{--emacs-dir=DIR}
@*@code{ESINGULAR_EMACS_DIR}
@tab Set the singular-emacs-home-directory, which is the directory where
singular.el can be found,  to @code{DIR}.

@item @code{--emacs-load=FILE}
@*@code{ESINGULAR_EMACS_LOAD}
@tab Load @code{FILE} on Emacs start-up, instead of the default load file.

@item @code{--singular=PROG}
@*@code{ESINGULAR_SINGULAR}
@tab Start @code{PROG} as @sc{Singular} program within Emacs

@end multitable

Notice that values of these options can also be given by setting the
above mentioned environment variables (where values  given as
command-line arguments take priority over values given by environment
variables).

@subheading Starting the interface within a running Emacs

If you are a more experienced Emacs user and you already have your own
local @code{.emacs} startup file, you might want to start the interface
out of your running Emacs without using @code{ESingular}. For this, you
should add the following lisp code to your @code{.emacs} file:

@smallexample
(setq load-path (cons "<singular-emacs-home-directory>" load-path))
(autoload 'singular "singular"
  "Start Singular using default values." t)
(autoload 'singular-other "singular"
  "Ask for arguments and start Singular." t)
@end smallexample

Then typing @code{M-x singular} in a running Emacs session initializes
the interface in a new buffer and launches a @sc{Singular} process. The
@sc{Singular} prompt comes up and you are ready to start your
@sc{Singular} session.

It is a good idea to take a look at the (well documented) file
@code{.emacs-singular} in the singular-emacs-home-directory, which comes
along with the distribution. In it you find some useful initializations
of the @sc{Singular} interface as well as some lisp code, which, for
example, adds a button to the XEmacs toolbar. Some of this code might be
useful for your @code{.emacs} file, too. And if you are a Emacs wizard,
it is of course a good idea to take a look at @code{singular.el} in the
singular-emacs-home-directory.

@subheading Starting, interrupting and stopping SINGULAR

There are the following commands to start and stop @sc{Singular}:

@itemize @bullet
@item @code{singular-other} (or menu @code{Singular}, item
@code{Start...})

Starts a @sc{Singular} process and asks for the following four
parameters in the minibuffer area:

@enumerate
@item
The Singular executable. This can either be a file name with complete
path, e.g., @code{/local/bin/Singular}. Then exactly this executable is
started. The path may contain the character @code{~} denoting your home
directory. Or it can be the name of a command without path, e.g.,
@code{Singular}. Then the executable is searched for in your
@code{$PATH} environment variable.

@item
The default working directory. This is the path to an existing
directory, e.g., @code{~/work}. The current directory is set to this
directory before @sc{Singular} is started.

@item
Command line options. You can set any @sc{Singular} command line
option (see @ref{Command line options}).

@item
The buffer name. You can specify the name of the buffer the interface is
running in.

@end enumerate

@item @code{singular} (or menu @code{Singular}, item @code{Start
default})

Starts @sc{Singular} with default settings for the executable, the
working directory, command line switches, and the buffer name. You can
customize this default settings, see
@ref{Customization of the Emacs interface}.

@item @code{singular-exit-singular} (bound to @code{C-c $} or menu
@code{Singular}, item @code{Exit})

Kills the running @sc{Singular} process of the current
buffer (but does not kill the buffer). Once you have killed a
@sc{Singular} process you can start a new one in the same buffer with
the command @code{singular} (or select the item @code{Start default} of
the @code{Singular} menu).

@item @code{singular-restart} (bound to @code{C-c C-r} or menu
@code{Singular}, item @code{Restart})

Kills the running @sc{Singular} process of the current buffer and starts
a new process in the same buffer with exactly the same command line
arguments as before.

@item @code{singular-control-c} (bound to @code{C-c C-c} or menu @code{Singular}, item @code{Interrupt})

Interrupt the Singular process running in the current buffer. Asks
whether to (@code{a})bort the current @sc{Singular} command,
(@code{q})uit or (@code{r})estart the current @sc{Singular} process, or
(@code{c})ontinue without doing anything (default).
@end itemize

Whenever a @sc{Singular} process is started within the Emacs interface,
the contents of a special startup file (by default
@code{~/.emacs-singularrc}) is pasted as input to @sc{Singular} at the
very end of the usual startup sequence (see @ref{Startup sequence}). The
name of the startup file can be changed, see
@ref{Customization of the Emacs interface}.


@c --------------------------------------------------------------------------
@node Demo mode, Customization of the Emacs interface, Running SINGULAR under Emacs, Emacs user interface
@subsection Demo mode
@cindex Emacs, Singular demo mode
@cindex Singular, demo mode
@cindex Demo mode

The Emacs interface can be used to run interactive @sc{Singular}
demonstrations. A demonstration is started by loading a so-called
@sc{Singular} demo file with the Emacs command
@code{singular-demo-load}, bound to
@code{C-c C-d}, or with the menu @code{Commands->Load Demo}.

A @sc{Singular} demo file should consist of @sc{Singular} commands
separated by blank lines. When running a demo, the input up to the next
blank line is echoed to the screen. Hitting @key{RETURN} executes the
echoed commands and shows their output. Hitting @key{RETURN} again,
echos the next commands to the screen, and so on, until all commands of
the demo file are executed. While running a demo, you can
execute other commands on the @sc{Singular} prompt: the next input of
from the demo file is then echoed again, if you hit @key{RETURN} on
an empty input line.

A @sc{Singular} demo can prematurely be exited by either starting
another demo, or by executing the Emacs command
@code{singular-demo-exit} (menu: @code{Commands->Exit Demo}).

Some aspects of running @sc{Singular} demos can be customized.
@xref{Customization of the Emacs interface}, for more info.


@c --------------------------------------------------------------------------
@node Customization of the Emacs interface, Editing SINGULAR input files with Emacs, Demo mode, Emacs user interface
@subsection Customization of the Emacs interface
@cindex Emacs, customization of Singular mode
@cindex Singular, customization of Emacs user interface
@cindex Customization of the Emacs interface

Emacs provides a convenient interface to customize the behavior of
Emacs and the SINGULAR Emacs interface for your own needs. You enter the
customize environment by either calling @code{M-x customize} (on XEmacs
you afterwards have to enter @code{emacs} in the minibuffer area) or by
selecting the menu item @code{Options->Customize->Emacs...} for XEmacs,
and the menu item @code{Help->Customize->Toplevel Customization Group}
for Emacs, resp. A brief introduction to the customization mode comes up
with the customization buffer. All customizable parameters are
hierarchically grouped and you can browse through all these groups and
change the values of the parameters using the mouse. At the end you can
safe the settings to a special file making your changes permanent.

To change the settings of the @sc{Singular} Emacs interface you can
either select the item @code{Preferences} of the @code{Singular} menu,
call @code{M-x customize-group} and give the argument
@code{singular-interactive} in the minibuffer area, or browse from the
top-level customization group through the path
@code{External->Singular->Singular interactive}.

The @sc{Singular} interface customization buffer is divided into four
groups:

@itemize @bullet
@item Singular Faces

Here you can specify various faces used if font-lock-mode is enabled
(which it is, by default).

@item Singular Sections And Foldings

Here you can specify special faces for @sc{Singular} input and output
and change the text used as replacement for folded sections.

For doing this, you also might
find handy the function @code{customize-face-at-point},
which let's you customize the face at the current position of
point. This function is automatically defined if you run
@code{ESingular}). Otherwise, you should add its definition (see below)
to your personal @code{.emacs} file.

@item Singular Interactive Miscellaneous

Here you can specify various different things such as the behavior of
the cursor keys, the name of the special @sc{Singular} startup file, the
appearance of the help window, or the default values for the
@code{singular} command.
@c Time: I find this sentence rather confusing
@c For a first look at customization of the
@c interface, this is probably the best group.


@item Singular Demo Mode

Here you can specify how chunks of the demo file are divided, or specify
a default directory for demo files.
@end itemize

When you run @code{ESingular}, the settings of customized variables are
saved in the file @code{$HOME/.emacs-singular-cust}. Otherwise, the
settings are appended to your @code{.emacs} file. Among others, this
means that the customized settings of @code{ESingular} are not
automatically  taken over by a "normal" Emacs, and vice versa.


@c --------------------------------------------------------------------------
@node Editing SINGULAR input files with Emacs, Top 20 Emacs commands, Customization of the Emacs interface, Emacs user interface
@subsection Editing SINGULAR input files with Emacs
@cindex Emacs, editing Singular input files
@cindex Singular, editing input files with Emacs
@cindex Editing SINGULAR input files with Emacs

Since @sc{Singular}'s programming language is similar to C, you
should use the Emacs C/C++-mode to edit @sc{Singular} input files and
@sc{Singular} libraries. Among others, this Emacs mode provides
automatic indentation, line-breaking and keyword highlighting.

When running @code{ESingular}, the C/C++-mode is automatically turned on
whenever a file with the suffix @code{.sing}, or @code{.lib} is loaded.

For Emacs sessions which were not started by @code{ESingular},  you
should add the following to your @code{.emacs} file:
@smallexample
;; turn on c++-mode for files ending in ".sing" and ".lib"
(setq auto-mode-alist (cons '("\\.sing\\'" . c++-mode) auto-mode-alist))
(setq auto-mode-alist (cons '("\\.lib\\'" .  c++-mode) auto-mode-alist))
;; turn-on fontification for c++-mode
(add-hook 'c++-mode-hook
         (function (lambda () (font-lock-mode 1))))
;; turn on aut-new line and hungry-delete
(add-hook 'c++-mode-hook
          (function (lambda () (c-toggle-auto-hungry-state 1))))
;; a handy function for customization
(defun customize-face-at-point ()
  "Customize face which point is at."
  (interactive)
  (let ((face (get-text-property (point) 'face)))
    (if face
        (customize-face face)
      (message "No face defined at point"))))

@end smallexample

Notice that you can change the default settings for source-code
highlighting (colors, fonts, etc.) by customizing the respective faces
using the @code{Customize} feature of Emacs. For doing this, you might
find handy the above given function @code{customize-face-at-point},
which lets you customize the face of the current position of point (this
function is automatically defined if you run @code{ESingular}).

@c --------------------------------------------------------------------------
@node Top 20 Emacs commands,,Editing SINGULAR input files with Emacs, Emacs user interface
@subsection Top 20 Emacs commands
@cindex Emacs, important commands
@cindex Singular, important commands of Emacs interface
@cindex Top 20 Emacs commands

Here is a list of the 20 probably most useful commands when using the
@sc{Singular} Emacs interface.

Starting and stopping of @sc{Singular}:

@itemize @bullet

@item @code{singular} (menu @code{Singular->Start Default...}): starts
@sc{Singular} using default arguments.

@item @code{singular-other} (menu @code{Singular->Start}): starts
@sc{Singular} asking for several arguments in the minibuffer area.

@item @code{singular-exit} (key @code{C-c $} or menu
@code{Singular->Exit}): kills the @sc{Singular} process running in the
current buffer (but does not kill the buffer).

@item @code{singular-restart} (key @code{C-c C-r} or menu
@code{Singular->Restart}): kills the @sc{Singular} process running in
the current buffer and starts a new @sc{Singular} process with exactly
the same arguments as before.

@end itemize

Editing input and output:

@itemize @bullet

@item @code{singular-beginning-of-line} (key @code{C-a}): moves point to
beginning of line, then skip past the @sc{Singular} prompt, if any.

@item @code{singular-toggle-truncate-lines} (key @code{C-c C-t} or menu
@code{Commands->Truncate lines}): toggles whether long lines should be
truncated or not. If lines are not truncated, the commands
@code{singular-scroll-left} and @code{singular-scroll-right} are useful
to scroll left and right, resp.

@item @code{singular-dynamic-complete} (key @code{TAB}): performs
context specific completion. If point is inside a string, file name
completion is done. If point is at the end of a help command (i.e.,
@code{help} or @code{?}), completion on @sc{Singular} help topics is
done. If point is at the end of an example command (i.e.,
@code{example}), completion is done on @sc{Singular} examples. In all
other cases, completion on @sc{Singular} commands is done.

@c note: add space between Fold/ Unfold to avoid overfull line
@item @code{singular-folding-toggle-fold-latest-output} (key @code{C-c
C-o} or menu @code{Commands->Fold/ Unfold Latest Output}): toggles
folding of the latest output section. If your last @sc{Singular} command
produced a huge output, simply type @code{C-c C-o} and it will be
replaced by a single line.

@item @code{singular-folding-toggle-fold-at-point} (key @code{C-c C-f}
or menu @code{Commands->Fold/Unfold At Point}): toggles folding of the
section the point currently is in.

@item @code{singular-folding-fold-all-output} (menu @code{Commands->Fold
All Output}): folds all @sc{Singular} output, replacing each output
section by a single line.

@item @code{singular-folding-unfold-all-output} (menu
@code{Commands->Unfold All Output}): unfolds all @sc{Singular} output
sections showing their true contents.

@end itemize

Loading of files and @sc{Singular} demo mode:

@itemize @bullet

@item @code{singular-load-library} (key @code{C-c C-l} or menu
@code{Commands->Libraries->other...}): asks for a standard library name
or a library file in the minibuffer (hit @code{TAB} for completion) and
loads the library into @sc{Singular}. The submenu @code{Libraries} of
the @code{Commands} menu also provides a separate menu item for each
standard library.

@item @code{singular-load-file} (key @code{C-c <} or menu
@code{Commands->Load File...}): asks for a file name in the minibuffer
(which is expanded using @code{expand-file-name} if given a prefix
argument) and loads the file into @sc{Singular}.

@item @code{singular-demo-load} (key @code{C-c C-d} or menu
@code{Commands->Load Demo...}): asks for a file name of a @sc{Singular}
demo file in the minibuffer area (hit @code{SPACE} for completion) and
enters the @sc{Singular} demo mode showing the first chunk of the demo.

@item @code{singular-demo-exit} (menu @code{Commands->Exit Demo}):
exits from @sc{Singular} demo mode and cleans up everything that is
left from the demo.

@end itemize

Help and Customization:

@itemize @bullet

@item @code{singular-help} (key @code{C-h C-s} or menu
@code{Singular->Singular Help}): asks for a @sc{Singular} help topic in
the minibuffer (hit @code{TAB} for completion) and shows the help text
in a separate buffer.

@item @code{singular-example} (key @code{C-c C-e} or menu
@code{Singular->Singular Example}): asks for a @sc{Singular} command in
the minibuffer (hit @code{TAB} for completion) and executes the example
of this command in the current @sc{Singular} buffer.

@item @code{customize-group} (menu @code{Singular->Preferences}): enters
the customization group of the @sc{Singular} Emacs interface. (If called
via @code{M-x customize-group} give argument @code{singular-interactive}
in the minibuffer area.)

@end itemize


@c ---------------------------------------------------------------------------
@c ---------------------------------------------------------------------------
@node Rings and orderings, Implemented algorithms, Emacs user interface, General concepts
@section Rings and orderings
@cindex Rings and orderings

All non-trivial algorithms in @sc{Singular} require the prior definition of
a ring. Such a ring can be

@enumerate
@item
a polynomial ring over a field,
@item
a localization of a polynomial ring,
@item
a quotient ring by an ideal of one of 1.@: or 2.,
@item
a tensor product of one of 1.@: or 2.
@end enumerate

Except for quotient rings, all of these rings are realized by choosing a
coefficient field, ring variables, and an appropriate global or local
monomial ordering on the ring variables.  @xref{Term orderings},
@ref{Mathematical background}.

The coefficient field of the rings may be

@cindex real
@cindex complex
@enumerate
@item
the field of rational numbers 
@ifinfo
@math{Q}
@end ifinfo
@tex
$Q$
@end tex
,
@item
@tex
finite fields $Z/p$, $p$ a prime $\le 2147483629$,
@end tex
@ifinfo
finite fields Z/p, p a prime <= 2147483629,
@end ifinfo
@item
@tex
finite fields $\hbox{GF}(p^n)$ with $p^n$ elements, $p$ a prime, $p^n \le 2^{15}$,
@end tex
@ifinfo
finite fields GF(p^n) with p^n elements, p a prime, p^n <= 2^15,
@end ifinfo
@item
transcendental extension of 
@ifinfo
@math{Q}
@end ifinfo
@tex
$Q$
@end tex
 or 
@ifinfo
@math{Z/p}
@end ifinfo
@tex
$Z/p$
@end tex
,
@item
simple algebraic extension of 
@ifinfo
@math{Q}
@end ifinfo
@tex
$Q$
@end tex
 or 
@ifinfo
@math{Z/p}
@end ifinfo
@tex
$Z/p$
@end tex
,
@item
the field of real numbers represented by floating point
numbers of a user defined precision.
@item
the field of complex numbers represented by (pairs of) floating point
numbers of a user defined precision.
@end enumerate

Throughout this manual, the current active ring in @sc{Singular} is
called basering.  The reserved name @code{basering} in @sc{Singular} is
an alias for the current active ring.  The basering can be set by
declaring a new ring as described in the following subsections or with
the commands @code{setring} and @code{keepring}.  @xref{keepring},
@ref{setring}.

Objects of ring dependent types are local to a ring. To access them
after a change of the basering they have to be mapped using @code{map} or by
the functions @code{imap} or @code{fetch}.  @xref{Objects}, @ref{fetch},
@ref{imap}, @ref{map}.

All changes of the basering in a procedure are local to this procedure
unless a @code{keepring} command is used as the last statement of the
procedure.  @xref{Procedures}, @ref{keepring}.

@menu
* Examples of ring declarations::
* General syntax of a ring declaration::
* Term orderings::
@end menu

@c --------------------------------------
@node Examples of ring declarations, General syntax of a ring declaration, Rings and orderings, Rings and orderings
@subsection Examples of ring declarations
@cindex Examples of ring declarations

The exact syntax of a ring declaration is given in the next two
subsections; this subsection lists some examples first. Note that the
ordering has to be chosen such that the unit-elements of the
ring are precisely those elements with leading monomial 1.  For more
information, see @ref{Monomial orderings}.

Every floating point number in a ring consists of two parts, which may be
chosen from the user. The leading part represents the number and the rest
is for the numerical stability. Two numbers with a difference only in the
rest are equal.

@itemize @bullet
@item
the ring 
@ifinfo
@math{Z/32003[x,y,z]}
@end ifinfo
@tex
$Z/32003[x,y,z]$
@end tex
 with degree reverse lexicographical
ordering.  The exact ring declaration may be omitted in the first
example since this is the default ring:

@smallexample
ring r;
ring r = 32003,(x,y,z),dp;
@end smallexample

@item
the ring 
@ifinfo
@math{Q[a,b,c,d]}
@end ifinfo
@tex
$Q[a,b,c,d]$
@end tex
 with lexicographical ordering:

@smallexample
ring r = 0,(a,b,c,d),lp;
@end smallexample

@item
the ring 
@ifinfo
@math{Z/7[x,y,z]}
@end ifinfo
@tex
$Z/7[x,y,z]$
@end tex
 with local degree reverse lexicographical
ordering.  The non-prime 10 is converted to the next lower prime in the
second example:

@smallexample
ring r = 7,(x,y,z),ds;
ring r = 10,(x,y,z),ds;
@end smallexample

@item
the ring
@tex
$Z/7[x_1,\ldots,x_6]$
@end tex
@ifinfo
Z/7[x1,@dots{},x6]
@end ifinfo
with lexicographical ordering for
@tex
$x_1,x_2,x_3$
@end tex
@ifinfo
x1,x2,x3
@end ifinfo
and degree reverse lexicographical ordering for
@tex
$x_4,x_5,x_6$:
@end tex
@ifinfo
x4,x5,x6:
@end ifinfo

@smallexample
ring r = 7,(x(1..6)),(lp(3),dp);
@end smallexample

@item
the localization of 
@ifinfo
@math{(Q[a,b,c])[x,y,z]}
@end ifinfo
@tex
$(Q[a,b,c])[x,y,z]$
@end tex
 at the maximal ideal

@ifinfo
@math{(x,y,z)}
@end ifinfo
@tex
$(x,y,z)$
@end tex
:

@smallexample
ring r = 0,(x,y,z,a,b,c),(ds(3), dp(3));
@end smallexample

@item
the ring 
@ifinfo
@math{Q[x,y,z]}
@end ifinfo
@tex
$Q[x,y,z]$
@end tex
 with weighted reverse lexicographical ordering.
The variables 
@ifinfo
@math{x}
@end ifinfo
@tex
$x$
@end tex
, 
@ifinfo
@math{y}
@end ifinfo
@tex
$y$
@end tex
, and 
@ifinfo
@math{z}
@end ifinfo
@tex
$z$
@end tex
 have the weights 2, 1,
and 3, respectively, and  vectors are first ordered by components (in
descending order) and then by monomials:
@smallexample
ring r = 0,(x,y,z),(c,wp(2,1,3));
@end smallexample
For ascending component order, the component ordering @code{C} has to be
used.

@item
the ring 
@ifinfo
@math{K[x,y,z]}
@end ifinfo
@tex
$K[x,y,z]$
@end tex
, where 
@ifinfo
@math{K=Z/7(a,b,c)}
@end ifinfo
@tex
$K=Z/7(a,b,c)$
@end tex
 denotes the transcendental
extension of 
@ifinfo
@math{Z/7}
@end ifinfo
@tex
$Z/7$
@end tex
 by 
@ifinfo
@math{a}
@end ifinfo
@tex
$a$
@end tex
, 
@ifinfo
@math{b}
@end ifinfo
@tex
$b$
@end tex
 and 
@ifinfo
@math{c}
@end ifinfo
@tex
$c$
@end tex
 with degree
lexicographical ordering:

@smallexample
ring r = (7,a,b,c),(x,y,z),Dp;
@end smallexample

@item
the ring 
@ifinfo
@math{K[x,y,z]}
@end ifinfo
@tex
$K[x,y,z]$
@end tex
, where 
@ifinfo
@math{K=Z/7[a]}
@end ifinfo
@tex
$K=Z/7[a]$
@end tex
 denotes the algebraic extension of
degree 2 of 
@ifinfo
@math{Z/7}
@end ifinfo
@tex
$Z/7$
@end tex
 by 
@ifinfo
@math{a.}
@end ifinfo
@tex
$a.$
@end tex
 In other words, 
@ifinfo
@math{K}
@end ifinfo
@tex
$K$
@end tex
 is the finite field with
49 elements.  In the first case, 
@ifinfo
@math{a}
@end ifinfo
@tex
$a$
@end tex
 denotes an algebraic
element over 
@ifinfo
@math{Z/7}
@end ifinfo
@tex
$Z/7$
@end tex
 with minimal polynomial
@tex
$\mu_a=a^2+a+3$,
@end tex
@ifinfo
mu_a=a^2+a+3,
@end ifinfo
in the second case, 
@ifinfo
@math{a}
@end ifinfo
@tex
$a$
@end tex

refers to some generator of the cyclic group of units of 
@ifinfo
@math{K}
@end ifinfo
@tex
$K$
@end tex
:

@smallexample
ring r = (7,a),(x,y,z),dp; minpoly = a^2+a+3;
ring r = (7^2,a),(x,y,z),dp;
@end smallexample

@item
the ring 
@ifinfo
@math{R[x,y,z]}
@end ifinfo
@tex
$R[x,y,z]$
@end tex
, where 
@ifinfo
@math{R}
@end ifinfo
@tex
$R$
@end tex
 denotes the field of real
numbers represented by simple precision floating point numbers. This is
a special case:

@smallexample
ring r = real,(x,y,z),dp;
@end smallexample

@item
the ring 
@ifinfo
@math{R[x,y,z]}
@end ifinfo
@tex
$R[x,y,z]$
@end tex
, where 
@ifinfo
@math{R}
@end ifinfo
@tex
$R$
@end tex
 denotes the field of real
numbers represented by floating point numbers of 50 valid decimal digits
and the same number of digits for the rest:

@smallexample
ring r = (real,50),(x,y,z),dp;
@end smallexample

@item
the ring 
@ifinfo
@math{R[x,y,z]}
@end ifinfo
@tex
$R[x,y,z]$
@end tex
, where 
@ifinfo
@math{R}
@end ifinfo
@tex
$R$
@end tex
 denotes the field of real
numbers represented by floating point numbers of 10 valid decimal digits
and with 50 digits for the rest:

@smallexample
ring r = (real,10,50),(x,y,z),dp;
@end smallexample

@item
the ring 
@ifinfo
@math{R(j)[x,y,z]}
@end ifinfo
@tex
$R(j)[x,y,z]$
@end tex
, where 
@ifinfo
@math{R}
@end ifinfo
@tex
$R$
@end tex
 denotes the field of real
numbers represented by floating point numbers of 30 valid decimal digits
and the same number for the rest. 
@ifinfo
@math{j}
@end ifinfo
@tex
$j$
@end tex
 denotes the imaginary unit.

@smallexample
ring r = (complex,30,j),(x,y,z),dp;
@end smallexample

@item
the ring 
@ifinfo
@math{R(i)[x,y,z]}
@end ifinfo
@tex
$R(i)[x,y,z]$
@end tex
, where 
@ifinfo
@math{R}
@end ifinfo
@tex
$R$
@end tex
 denotes the field of real
numbers represented by floating point numbers of 6 valid decimal digits
and the same number for the rest. 
@ifinfo
@math{i}
@end ifinfo
@tex
$i$
@end tex
 is the default for the imaginary unit.

@smallexample
ring r = complex,(x,y,z),dp;
@end smallexample

@item
the quotient ring 
@ifinfo
@math{Z/7[x,y,z]}
@end ifinfo
@tex
$Z/7[x,y,z]$
@end tex
 modulo the square of the maximal
ideal 
@ifinfo
@math{(x,y,z)}
@end ifinfo
@tex
$(x,y,z)$
@end tex
:

@smallexample
ring R = 7,(x,y,z), dp;
qring r = std(maxideal(2));
@end smallexample
@end itemize

@c ---------------------------------------------------------------------------
@node General syntax of a ring declaration, Term orderings, Examples of ring declarations, Rings and orderings
@subsection General syntax of a ring declaration
@cindex General syntax of a ring declaration

@subsubheading Rings
@c ------------------------------------------------------------
@c   This piece of text (partially) exists also in the
@c   file types.doc, chapter "ring declarations".
@c   If you change something here, change it there, too!
@c ------------------------------------------------------------
@table @strong
@item Syntax:
@code{ring} name @code{=} @code{(}coefficient_field@code{),}
  @code{(} names_of_ring_variables @code{),}
  @code{(} ordering @code{);}
@item Default:
@code{32003,(x,y,z),(dp,C);}
@item Purpose:
declares a ring and sets it as the current basering.
@end table

The coefficient_field is given by one of the following:
@enumerate
@item
a non-negative int_expression less or equal 32003.
@* The int_expression should either be 0, specifying the field of
rational numbers Q, or a prime number p, specifying the finite field
with p elements.  If it is not a prime number, int_expression is
converted to the next lower prime number.

@item
an expression_list of an int_expression and one or more names.
@* The int_expression specifies the characteristic of the coefficient
field as described above. The names are used as parameters in
transcendental or algebraic extensions of the coefficient
field. Algebraic extensions are implemented for one parameter only. In
this case, a minimal polynomial has to be defined by assignment to
@code{minpoly}. @xref{minpoly}.

@item
an expression_list of an int_expression and a name.
@* The int_expression has to be a prime number p to the power of a
positive integer n. This defines the Galois field
@tex
$\hbox{GF}(p^n)$ with $p^n$ elements, where $p^n$ has to be smaller or equal $2^{15}$.
@end tex
@ifinfo
GF(p^n) with p^n elements, where p^n has to be smaller or equal 2^15.
@end ifinfo
The given name refers to a primitive element of
@tex
$\hbox{GF}(p^n)$
@end tex
@ifinfo
GF(p^n)
@end ifinfo
generating the multiplicative group.  Due to a different internal
representation, the arithmetic operations in these coefficient fields
are faster than arithmetic operations in algebraic extensions as
described above.

@item
an expression_list of the name @code{real} and two optional int_expression
giving the precision in decimal digits and
the size for the stabilizing rest. The default for the rest is the same size
as for the representation.
An exception is the name @code{real} without any integers.
These numbers are implemented as machine floating point numbers
of single precision.
Note that computations over all these fields are not exact.

@item
an expression_list of the name @code{complex},
two optional int_expression and a name.
This specifies the field of complex numbers represented with floating point
numbers with a precision similar to @code{real}. An expression_list
without int_expression defines a precision and rest with length 6.
The name of the imaginary unit is given by the last parameter.
Note that computations over these fields are not exact.
@end enumerate

'names_of_ring_variables' is a list of names or indexed names.

'ordering' is a list of block orderings where each block ordering is either

@enumerate
@item
@code{lp}, @code{dp}, @code{Dp}, @code{ls}, @code{ds}, or @code{Ds}
optionally followed by a size parameter in parentheses.

@item
@code{wp}, @code{Wp}, @code{ws}, @code{Ws}, or @code{a} followed by a
weight vector given as an intvec_expression in parentheses.

@item
@code{M} followed by an intmat_expression in parentheses.

@item
@code{c} or @code{C}.
@end enumerate

For the definition of the orderings, see @ref{Monomial orderings}.

If one of coefficient_field, names_of_ring_variables, and ordering
consists of only one entry, the parentheses around this entry may be
omitted.
@c ------------------------------------------------------------
@c   End of duplicate piece of text.
@c ------------------------------------------------------------

@subsubheading Quotient rings
@c ------------------------------------------------------------
@c   This piece of text (partially) exists also in the
@c   file types.doc, chapter "qring declarations", and in
@c   singular.doc, chapter "Miscellaneous oddities".
@c   If you change something here, change it there, too!
@c ------------------------------------------------------------
@table @strong
@item Syntax:
@code{qring} name @code{=} ideal_expression @code{;}
@item Default:
none
@item Purpose:
declares a quotient ring as the basering modulo ideal_expression.  Sets
it as current basering.
@end table

ideal_expression has to be represented by a standard basis.

The most convenient way to map objects from a ring to its quotient ring
and vice versa is to use the @code{fetch} function (@pxref{fetch}).

@sc{Singular} computes in a quotient ring as long as possible with the
given representative of a polynomial, say, @code{f}.  I.e., it usually
does not reduce @code{f} w.r.t. the quotient ideal.  This is only done
when necessary
during standard bases computations or by an explicit reduction using
the command @code{reduce(f, std(0))} (@pxref{reduce}).

@strong{Example:}
@smallexample
@c computed example Quotient_rings general.doc:1784 
  ring r=32003,(x,y),dp;
  poly f=x3+yx2+3y+4;
  qring q=std(maxideal(2));
  basering;
@expansion{} //   characteristic : 32003
@expansion{} //   number of vars : 2
@expansion{} //        block   1 : ordering dp
@expansion{} //                  : names    x y 
@expansion{} //        block   2 : ordering C
@expansion{} // quotient ring from ideal
@expansion{} _[1]=y2
@expansion{} _[2]=xy
@expansion{} _[3]=x2
  poly g=fetch(r, f);
  g;
@expansion{} x3+x2y+3y+4
  reduce(g,std(0));
@expansion{} 3y+4
@c end example Quotient_rings general.doc:1784
@end smallexample

@c ------------------------------------------------------------
@c   End of duplicate piece of text.
@c ------------------------------------------------------------

@c ---------------------------------------------------------------------------
@node Term orderings,  , General syntax of a ring declaration, Rings and orderings
@subsection Term orderings
@cindex Monomial orderings, Term orderings

Any polynomial (resp.@: vector) in @sc{Singular} is ordered w.r.t. a
term ordering (or, monomial ordering), which has to be specified
together with the declaration of a ring. @sc{Singular} stores and
displays a polynomial (resp.@: vector) w.r.t. this ordering, i.e., the
greatest monomial (also called the leading monomial) is the first one
appearing in the output polynomial, and the smallest monomial is the last one.

@strong{Remark:} The novice user should generally use the ordering
@code{dp} for computations in the polynomial ring
@tex
$K[x_1,\ldots,x_n]$,
@end tex
@ifinfo
K[x1,...,xn],
@end ifinfo
resp.@:  @code{ds} for computations in the localization
@tex
$\hbox{Loc}_{(x)}K[x_1,\ldots,x_n])$.
@end tex
@ifinfo
Loc_(x)K[x1,...,xn]).
@end ifinfo
For more details, see @ref{Polynomial data}.

In a ring declaration, @sc{Singular} offers the following orderings:
@enumerate
@item Global orderings
@table @asis
@item @code{lp}
lexicographical ordering

@item @code{dp}
degree reverse lexicographical ordering

@item @code{Dp}
degree lexicographical ordering

@item @code{wp(} intvec_expression @code{)}
weighted reverse lexicographical ordering; the weight vector may
consist of positive integers only.

@item @code{Wp(} intvec_expression @code{)}
weighted lexicographical ordering; the weight vector may consist of
positive integers only.
@end table

Global orderings are well-orderings, i.e., 
@ifinfo
@math{1 < x}
@end ifinfo
@tex
$1 < x$
@end tex
 for each ring
variable 
@ifinfo
@math{x}
@end ifinfo
@tex
$x$
@end tex
. They are denoted by a @code{p} as the second
character in their name.

@item Local orderings
@table @asis
@item @code{ls}
negative lexicographical ordering

@item @code{ds}
negative degree reverse lexicographical ordering

@item @code{Ds}
negative degree lexicographical ordering

@item @code{ws(} intvec_expression @code{)}
(general) weighted reverse lexicographical ordering; the first element
of the weight vector has to be non-zero.

@item @code{Ws(} intvec_expression @code{)}
(general) weighted lexicographical ordering; the first element
of the weight vector has to be non-zero.
@end table

Local orderings are not well-orderings. They are denoted by an @code{s}
as the second character in their name.

@item Matrix orderings
@table @asis
@item @code{M(} intmat_expression @code{)}
intmat_expression has to be an invertible square matrix
@end table

Using matrix orderings, @sc{Singular} can compute standard bases
w.r.t.@: any monomial ordering that is compatible with the natural
semi-group structure on the monomials. In practice, the predefined global
and local orderings together with the block orderings should be
sufficient in most cases. These orderings are faster than their
corresponding matrix orderings since evaluation of a matrix ordering is
time consuming.

@item Extra weight vector
@table @asis
@item @code{a(} intvec_expression @code{)}
an extra weight vector @code{a(} intvec_expression @code{)} may precede
any monomial ordering
@end table

@item Product ordering
@table @asis
@item @code{(} ordering [ @code{(} int_expression @code{)} ]@code{,} @dots{} @code{)}
any of the above orderings and the extra weight vector may be combined
to yield product or block orderings
@end table

The orderings @code{lp}, @code{dp}, @code{Dp}, @code{ls}, @code{ds}, and
@code{Ds} may be followed by an int_expression in parentheses giving the
size of the block. For the last block the size is calculated
automatically. For the weighted orderings the size of the block is given
by the size of the weight vector. The same holds analogously for matrix
orderings.

@item Module orderings
@table @asis
@itemx @code{(} ordering@code{,} @dots{}@code{, C )}
@item @code{(} ordering@code{,} @dots{}@code{, c )}
sort polynomial vectors by the monomial ordering first, then by
components

@itemx @code{( C,} ordering@code{,} @dots{} @code{)}
@item @code{( c,} ordering@code{,} @dots{} @code{)}
sort polynomial vectors by components first, then by the monomial
ordering
@end table

Here a capital @code{C} sorts generators in ascending order, i.e.,
@code{gen(1)} < @code{gen(2)} < @enddots{} A small @code{c} sorts in
descending order, i.e., @code{gen(1)} > @code{gen(2)} > @enddots{} It is
not necessary to specify the module ordering explicitly since @code{(}
ordering@code{,} @dots{}@code{, C )} is the default.

In fact, @code{c} or @code{C} may be specified anywhere in a product
ordering specification, not only at its beginning or end.  All monomial
block orderings preceding the component ordering have higher precedence,
all monomial block orderings following after it have lower precedence.
@end enumerate

For a mathematical description of these orderings, see
@ref{Polynomial data}.

@c ---------------------------------------------------------------------------
@node Implemented algorithms, The SINGULAR language, Rings and orderings, General concepts
@section Implemented algorithms
@cindex Implemented algorithms

The basic algorithm in @sc{Singular} is a general standard basis algorithm for
any monomial ordering which is compatible with the natural semi-group
structure of the exponents. This includes well-orderings
(Buchberger algorithm to compute a Groebner basis)
and tangent cone orderings (Mora algorithm) as special cases.

Nonetheless, there are a lot of other important algorithms:
@itemize @bullet

@item
Algorithms to compute the standard operations on ideals and modules:
intersection, ideal quotient, elimination, etc.

@item
Different Syzygy algorithms and algorithms to compute free resolutions
of modules.

@item
Combinatorial algorithms to compute dimensions, Hilbert series,
multiplicities, etc.

@item
Algorithms for univariate and multivariate polynomial factorization,
resultant and gcd computations.
@end itemize

@c ----------------------------------------------------------------------
@subheading Commands to compute standard bases
@table @code
@item facstd
@ref{facstd}
@*computes a list of Groebner bases via the Factorizing Groebner Basis
Algorithm, i.e., their intersection has the same radical as the original ideal. It need not
be a Groebner basis of the given ideal.

The intersection of the zero-sets is the zero-set of the given ideal.
@item fglm
@ref{fglm}
@*computes a Groebner basis provided that a reduced Groebner basis
w.r.t.@: another ordering is given.

Implements the so-called FGLM (Faugere, Gianni, Lazard, Mora) algorithm.
The given ideal must be zero-dimensional.
@item groebner
@ref{groebner}@*
computes a standard resp.@: Groebner bases using a heuristically chosen
method.

This is the preferred method to compute a standard resp.@: Groebner
bases.
@item mstd
@ref{mstd}
@*computes a standard basis and a minimal set of generators.
@item std
@ref{std}
@*computes a standard resp.@: Groebner basis.
@item stdfglm
@ref{stdfglm}
@*computes a Groebner basis in a ring with a ``difficult'' ordering
(e.g., lexicographical) via @code{std} w.r.t.@: a ``simple'' ordering and
@code{fglm}.

The given ideal must be zero-dimensional.
@item stdhilb
@ref{stdhilb}@*
computes a Groebner basis in a ring with a ``difficult'' ordering
(e.g., lexicographical) via @code{std} w.r.t.@: a ``simple'' ordering
and a @code{std} computation guided by the Hilbert series.
@end table

@c ----------------------------------------------------------------------
@subheading Further processing of standard bases
The next commands require the input to be a standard basis.

@table @code
@item degree
@ref{degree}
@*computes the (Krull) dimension, codimension and the multiplicity.

The result is only displayed on the screen.
@item dim
@ref{dim}@*
computes the dimension of the ideal resp.@: module.
@item highcorner
@ref{highcorner}@*
computes the smallest monomial not contained in the ideal resp.@: module.
The ideal resp.@: module has to be finite dimensional as a vector space
over the ground field.
@item hilb
@ref{hilb}@*
computes the first, and resp.@: or, second Hilbert series of an ideal
resp.@: module.
@item kbase
@ref{kbase}@* computes a vector space basis (consisting of monomials) of
the quotient of a ring by an ideal resp.@: of a free module by a
submodule.

The ideal resp.@: module has to be finite dimensional as a vector space
over the ground field and has to be represented by a standard basis w.r.t.@:
the ring ordering.
@item mult
@ref{mult}@*
computes the degree of the monomial ideal resp.@: module generated by
the leading monomials of the input.
@item reduce
@ref{reduce}@*
reduces a polynomial, vector, ideal or module to its normal form with
respect to an ideal or module represented by a standard basis.
@item vdim
@ref{vdim}@*
computes the vector space dimension of a ring (resp.@: free module)
modulo an ideal (resp.@: module).
@end table

@c ----------------------------------------------------------------------
@subheading Commands to compute resolutions
@table @code
@item res
@ref{res}@*
computes a free resolution of an ideal or module using a heuristically
chosen method.
This is the preferred method to compute free resolutions of ideals or
modules.
@item lres
@ref{lres}@*
computes a free resolution of an ideal or module with La Scala's
method. The input needs to be homogeneous.
@item mres
@ref{mres}@*
computes a minimal free resolution of an ideal or module with the Syzygy
method.
@item sres
@ref{sres}@*
computes a free resolution of an ideal or module with Schreyer's
method. The input has to be a standard basis.
@item nres
@ref{nres}@*
computes a free resolution of an ideal or module with the standard basis
method.
@item syz
@ref{syz}@*
computes the first Syzygy (i.e., the module of relations of the
given generators).
@end table

@c ----------------------------------------------------------------------
@subheading Further processing of resolutions
@table @code
@item betti
@ref{betti}@*
computes the graded Betti numbers of a module from a free resolution.
@item minres
@ref{minres}@*
minimizes a free resolution of an ideal or module.
@item regularity
@ref{regularity}@*
computes the regularity of a homogeneous ideal resp.@: module from a given
minimal free resolution.
@end table

@c ----------------------------------------------------------------------
@subheading Processing of polynomials
@table @code
@item char_series
@ref{char_series}@*
computes characteristic sets of polynomial ideals.
@item extgcd
@ref{extgcd}@*
computes the extended gcd of two polynomials.

Implemented as extended Euclidean Algorithm.  Applicable for univariate
polynomials only.
@item factorize
@ref{factorize}@*
computes factorization of univariate and multivariate polynomials into
irreducible factors.

The most basic algorithm is univariate factorization in prime
characteristic.  The Cantor-Zassenhaus Algorithm is used in this case.
For characteristic 0, a univariate Hensel-lifting is done to lift from
prime characteristic to characteristic 0.  For multivariate
factorization in any characteristic, the problem is reduced to the
univariate case first, then a multivariate Hensel-lifting is used to
lift the univariate factorization.

Note that there is no factorization of polynomials over algebraic
extensions of Q.
@item gcd
@ref{gcd}@*
computes greatest common divisors of univariate and multivariate polynomials.

For prime characteristic, a subresultant gcd is used.  In characteristic 0, a
modular algorithm is used for the univariate case.  For the multivariate case,
the EZGCD is used.

Note that there is no gcd calculation for polynomials over algebraic
extensions of Q.
@item resultant
@ref{resultant}@*
computes the resultant of two univariate polynomials using the subresultant
algorithm.

Multivariate polynomials are considered as univariate polynomials in the
main variable (which has to be specified by the user).
@item vandermonde
@ref{vandermonde}@*
interpolates a polynomial from its values at several points
@end table

@c ----------------------------------------------------------------------
@subheading Matrix computations
@table @code
@item bareiss
@ref{bareiss}@*
implements sparse Gauss-Bareiss method for elimination (matrix
triangularization) in arbitrary integral domains.
@item det
@ref{det}@*
computes the determinant of a square matrix.

For matrices with integer entries a modular algorithm is used.  For
other domains the Gauss-Bareiss method is used.
@item minor
@ref{minor}@*
computes all minors (=subdeterminants) of a given size for a matrix.
@end table

@c ----------------------------------------------------------------------
@subheading Numeric computations
@table @code
@item laguerre
@ref{laguerre}@*
computes all (complex) roots of a univariate polynomial
@item uressolve
@ref{uressolve}@*
find all roots of 0-dimensional ideal i with multivariate resultants
@end table

@c ----------------------------------------------------------------------
@subheading Controlling computations
@table @code
@item option
@ref{option}@*
allows setting of options for manipulating the behavior of computations
(such as reduction strategies) and for showing protocol information indicating
the progress of a computation.
@end table

@c ---------------------------------------------------------------------------
@node The SINGULAR language, Input and output, Implemented algorithms, General concepts
@section The SINGULAR language
@cindex The SINGULAR language

@sc{Singular} interprets commands given interactively on the command
line as well as given in the context of user-defined procedures.  In
fact, @sc{Singular} makes no distinction between these two cases.
Thus, @sc{Singular} offers a powerful programming language as well as
an easy-to-use command line interface without differences in syntax or
semantics.

In the following, the basic language concepts such as commands,
expressions, names, objects, etc., are discussed.  @xref{Procedures},
and @ref{Libraries},
for the concepts of procedures
and libraries.

In many aspects, the @sc{Singular} language is similar to the C
programming language.  For a description of some of the subtle
differences, see @ref{Major differences to the C programming language}.

@subsubheading Elements of the language

The major building blocks of the @sc{Singular} language  are
expressions, commands, and control structures.  The notion of expressions
in the @sc{Singular} and the C programming language are identical,
whereas the notion of commands and control structures only roughly
corresponds to the C statements.

@itemize @bullet

@item
An ``expression'' is a sequence of operators, functions, and operands
that specifies a computation.  An expression always results in a
value of a specific type.  @xref{Data types}, and its subsections (e.g.,
@ref{poly expressions}), for information on how to build expressions.

@item
A ``command'' is either a declaration, an assignment, a call to a
function without return value, or a print command.  For detailed
information, see @ref{General command syntax}.

@item
``Control structures'' determine the execution sequence of commands.
@sc{Singular} provides control structures for conditional execution
(@code{if} @dots{} @code{else}) and iteration (@code{for} and
@code{while}).  Commands may be grouped in pairs of @code{@{} @code{@}}
(curly brackets) to form blocks.  @xref{Control structures}, for
more information.
@end itemize

@subsubheading Other notational conventions
For user-defined functions, the notion of ``procedure'' and
``function'' are synonymous.

As already mentioned above, functions without return values are called
commands.  Furthermore, whenever convenient, the term ``command'' is
used for a function, even if it does return a value.

@menu
* General command syntax::
* Special characters::
* Names::
* Objects::
* Type conversion and casting::
* Flow control::
@end menu

@c ---------------------------------------------------------------------------
@node General command syntax, Special characters, The SINGULAR language, The SINGULAR language
@subsection General command syntax
@cindex General command syntax

In @sc{Singular} a command is either a declaration, an assignment, a
call to a function without return value, or a print command. The general
form of a command is described in the following subsections.

@subsubheading Declaration
@enumerate
@item type name @code{=} expression @code{;}
@* declares a variable with the given name of the given type and assigns
the expression as initial value to it.  Expression is an expression of
the specified type or one that can be converted to that type.
@xref{Type conversion and casting}.

@item type name_list @code{=} expression_list @code{;}
@* declares variables with the given names and assigns successively each
expression of expression_list to the corresponding name of
name_list.  Both lists must be of the same length.  Each expression in
expression_list is an expression of the specified type or one that can
be converted to that type.  @xref{Type conversion and casting}.

@item type name @code{;}
@* declares a variable with the given name of the given type and assigns
the default value of the specific type to it.
@end enumerate

@xref{Names}, for more information on declarations.  @xref{Data types},
for a description of all data types known to @sc{Singular}.

@smallexample
ring r;                   // the default ring
poly f,g = x^2+y^3,xy+z2; // the polynomials f=x^2+y^3 and g=x*y+z^2
ideal I = f,g;            // the ideal generated by f and g
matrix m[3][3];           // a 3 x 3 zero matrix
int i=2;                  // the integer i=2
@end smallexample

@subsubheading Assignment
@enumerate 4
@item
name @code{=} expression @code{;}
@* assigns expression to name.

@item
name_list @code{=} expression_list @code{;}
@* assigns successively each expression of expression_list to the
corresponding name of name_list. Both lists must be of the same
length.  This is not a simultaneous assignment.  Thus, @code{f, g = g, f;} does
not swap the values of @code{f} and @code{g}, but rather assigns
@code{g} to both @code{f} and @code{g}.
@end enumerate

There must be a type conversion of the type of expression to the type of
name. @xref{Type conversion and casting}.

An assignment itself does not yield a value. Hence, compound assignments
like @code{i = j = k;} are not allowed and result in an error.

@smallexample
f = x^2 + y^2 ;      // overwrites the old value of f
I = jacob(f);
f,g = I[1],x^2+y^2 ; // overwrites the old values of f and g
@end smallexample

@subsubheading Function without return value
@enumerate 6
@item
function_name [ @code{(} argument_list @code{)} ] @code{;}
@* calls function function_name with arguments argument_list.
@end enumerate

The function may have output (not to be confused with a return value of
type string). @xref{Functions}. Functions without a return value are
specified there to have a return type 'none'.

Some of these functions have to be called without parentheses, e.g.,
@code{help}, @code{LIB}.

@smallexample
@c computed example Function_without_return_value general.doc:2344 
  ring r;
  ideal i=x2+y2,x;
  i=std(i);
  degree(i);        // degree has no return value but prints output
@expansion{} 2
@c end example Function_without_return_value general.doc:2344
@end smallexample

@subsubheading Print command
@enumerate 7
@item
expression @code{;}
@* prints the value of an expression, for example, of a variable.
@end enumerate

Use the function @code{print} (or the procedure show from inout.lib)
to get a pretty output of various data
types, e.g., matrix or intmat. @xref{print}.

@smallexample
@c computed example Print_command general.doc:2364 
  int i=2;
  i;
@expansion{} 2
  intmat m[2][2]=1,7,10,0;
  print(m);
@expansion{}      1     7
@expansion{}     10     0
@c end example Print_command general.doc:2364
@end smallexample

@c ------------------------------------------------------------
@node Special characters, Names, General command syntax, The SINGULAR language
@subsection Special characters
@cindex Special characters
The following characters and operators have special meaning:
@table @asis
@item @code{=}
@cindex =
assignment
@item @code{(}, @code{)}
@cindex (
@cindex )
@cindex indexed names
in expressions, for indexed names and for argument lists
@item @code{[}, @code{]}
@cindex [
@cindex ]
access operator for strings, integer vectors, ideals, matrices,
polynomials, resolutions, and lists.  Used to build vectors of
polynomials.  Example: @code{s[3]}, @code{m[1,3]}, @code{i[1..3]},
@code{[f,g+x,0,0,1]}.
@item @code{+}
@cindex +
addition operator
@item @code{-}
@cindex -
subtraction operator
@item @code{*}
@cindex *
multiplication operator
@item @code{/}
@cindex /
division  operator.  @xref{Miscellaneous oddities}, for the difference
between the division operators @code{/} and @code{div}.
@item @code{%}
@cindex %
modulo operator
@item @code{^} or @code{**}
@cindex ^
@cindex **
exponentiation operator
@item @code{==}
@cindex ==
comparison operator equal
@item @code{!=} or @code{<>}
@cindex !=
@cindex <>
comparison operator not equal
@item @code{>=}
@cindex >=
comparison operator bigger or equal
@item @code{>}
@cindex >
comparison operator bigger
@item @code{<=}
@cindex <=
comparison operator smaller or equal
@item @code{<}
@cindex <
comparison operator smaller.  Also used for file input.  @xref{filecmd}.
@item @code{!}
@cindex !
boolean operator not
@item @code{&&}
@cindex &&
boolean operator and
@item @code{||}
@cindex ||
boolean operator or
@item @code{"}
@cindex "
delimiter for string constants
@item @code{`}
@cindex `
delimiter for name substitution
@item @code{?}
@cindex ?
synonym for @code{help}
@item @code{//}
@cindex //
comment delimiter.  Comment extends to end of line.
@item @code{/*}
@cindex //
comment delimiter.  Starts a comment which ends with @code{*/}.
@item @code{*/}
@cindex //
comment delimiter.  Ends a comment which starts with @code{/*}.
@item @code{;}
@cindex ;
statement separator
@item @code{,}
separator for expression lists and function arguments
@item @code{\}
@cindex \
escape character for @code{"} and @code{\} within strings
@item @code{..}
@cindex ..
interval specifier returning intvec.
E.g., @code{1..3} which is equivalent to the intvec @code{1, 2, 3}.
@item @code{_}
@cindex _
value of expression last displayed
@item @code{~}
@cindex ~
breakpoint in procedures
@item @code{#}
@cindex #
list of parameters in procedures without explicit parameter list
@item @code{$}
@cindex $
terminates @sc{Singular}
@end table

@c ------------------------------------------------------------
@node Names, Objects, Special characters, The SINGULAR language
@subsection Names
@cindex Names
@cindex Identifiers, syntax of

@sc{Singular} is a strongly typed language.  This means that all names
(= identifiers) have to be declared prior to their use.  For the general
syntax of a declaration, see the description of declaration commands
(@pxref{General command syntax}).

@xref{Data types}, for a description of @sc{Singular}'s data types.
@xref{typeof}, for a short overview of possible types.  To get information
on a name and the object named by it, the @code{type} command may be
used (@pxref{type}).

It is possible to redefine an already existing name if doing so does not
change its type. A redefinition first sets the variable to the default
value and then computes the expression. The difference between
redefining and overwriting a variable is shown in the following example:

@smallexample
@c computed example Names general.doc:2507 
  int i=3;
  i=i+1;        // overwriting
  i;
@expansion{} 4
  int i=i+1;    // redefinition
@expansion{} // ** redefining i **
  i;
@expansion{} 1
@c end example Names general.doc:2507
@end smallexample

User defined names should start with a letter and consist of letters and
digits only.  As an exception to this rule, the characters @code{@@},
and @code{_} may
be used as part of a name, too.  Capital and small letters are
distinguished.  Indexed names are built as a name followed by an
int_expression in parentheses.  A list of indexed names can be built as
a name followed by an intvec_expression in parentheses.

@smallexample
@c computed example Names_1 general.doc:2525 
  ring R;
  int n=3;
  ideal j(3);
  ideal j(n);     // is equivalent to the above
@expansion{} // ** redefining j(3) **
  ideal j(2)=x;
  j(2..3);
@expansion{} j(2)[1]=x j(3)[1]=0
@c end example Names_1 general.doc:2525
@end smallexample

Names may not coincide with reserved names (keywords).  Type
@code{reservedName();} to get a list of the reserved names.
@xref{reservedName}.  Names should not interfere with names of ring
variables or, more generally, with monomials.  @xref{Identifier
resolution}.
@* The command @code{listvar} provides a list of the names in use
(@pxref{listvar}).

The most recently printed expression is available
under the special name @code{_}, e.g.,
@smallexample
@c computed example Names_2 general.doc:2546 
  ring r;
  ideal i=x2+y3,y3+z4;
  std(i);
@expansion{} _[1]=y3+x2
@expansion{} _[2]=z4-x2
  ideal k=_;
  k*k+x;
@expansion{} _[1]=y6+2x2y3+x4
@expansion{} _[2]=y3z4+x2z4-x2y3-x4
@expansion{} _[3]=z8-2x2z4+x4
@expansion{} _[4]=x
  size(_[3]);
@expansion{} 3
@c end example Names_2 general.doc:2546
@end smallexample

A string_expression enclosed in @code{`}@dots{}@code{`} (back ticks)
evaluates to the value of the variable named by the string_expression.
This feature is referred to as name substitution.

@smallexample
@c computed example Names_3 general.doc:2561 
  int foo(1)=42;
  string bar="foo";
  `bar+"(1)"`;
@expansion{} 42
@c end example Names_3 general.doc:2561
@end smallexample

@c ------------------------------------------------------------
@node Objects, Type conversion and casting, Names, The SINGULAR language
@subsection Objects
@cindex Objects

Every object in @sc{Singular} has a type and a value.  In most cases it
has also a name and in some cases an attribute list.  The value of an
object may be examined simply by printing it with a print command:
object@code{;}.  The type of an object may be determined by means of the
@code{typeof} function, the attributes by means of the @code{attrib}
function (@ref{typeof}, @ref{attrib}):

@smallexample
@c computed example Objects general.doc:2581 
  ring r=0,x,dp;
  typeof(10);
@expansion{} int
  typeof(10000000000000000);
@expansion{} number
  typeof(r);
@expansion{} ring
  attrib(x);
@expansion{} no attributes
  attrib(std(ideal(x)));
@expansion{} attr:isSB, type int
@c end example Objects general.doc:2581
@end smallexample

Each object of type
@code{poly},
@code{ideal},
@code{vector},
@code{module},
@code{map},
@code{matrix},
@code{number}, or
@code{resolution}
belongs to a specific ring.  Also @code{list}, if at least one of the
objects contained in the list belongs to a ring.  These objects are local
to the ring.  Their names can be used for other objects in other rings.
Objects from one ring can be mapped to another ring using maps or with the
commands @code{fetch} or @code{imap}.  @xref{map}, @ref{fetch}, @ref{imap}.

All other types do not belong to a ring and can be accessed within every
ring and across rings.  They can be declared even if there is no active
basering.

@c ------------------------------------------------------------
@node Type conversion and casting, Flow control, Objects, The SINGULAR language
@subsection Type conversion and casting

@subsubheading Type conversion
@cindex Type conversion
Assignments convert the type of the right-hand side to the type of the
left-hand side of the assignment, if possible.  Operators and functions
which require certain types of operands can also implicitly convert the
type of an expression.  It is, for example, possible to multiply a
polynomial by an integer because the integer is automatically converted to
a polynomial.  Type conversions do not act transitively.  Possible
conversions are:


@c ------------------------------------------------
@c Do not delete the leading `@ ' for indentation of the whole table.  The
@c table is indented since regular @tables are indented by a small amount,
@c too.
@c ------------------------------------------------
@multitable @columnfractions .05 .18 .81
@item
@   1. @tab @code{int}  @tab @expansion{} @code{ideal}
@item
@   2. @tab @code{poly}  @tab @expansion{} @code{ideal}
@item
@   3. @tab @code{intvec}  @tab @expansion{} @code{intmat}
@item
@   4. @tab @code{int}  @tab @expansion{} @code{intvec}
@item
@   5. @tab @code{int}  @tab @expansion{} @code{intmat}
@item
@   6. @tab @code{string}  @tab @expansion{} @code{link}
@item
@   7. @tab @code{resolution} @tab @expansion{} @code{list}
@item
@   8. @tab @code{ideal}  @tab @expansion{} @code{matrix}
@item
@   9. @tab @code{int}  @tab @expansion{} @code{matrix}
@item
@  10. @tab @code{intmat}  @tab @expansion{} @code{matrix}
@item
@  11. @tab @code{intvec}  @tab @expansion{} @code{matrix}
@item
@  12. @tab @code{module}  @tab @expansion{} @code{matrix}
@item
@  13. @tab @code{number}  @tab @expansion{} @code{matrix}
@item
@  14. @tab @code{poly}  @tab @expansion{} @code{matrix}
@item
@  15. @tab @code{vector}  @tab @expansion{} @code{matrix}
@item
@  16. @tab @code{ideal}  @tab @expansion{} @code{module}
@item
@  17. @tab @code{matrix}  @tab @expansion{} @code{module}
@item
@  18. @tab @code{vector}  @tab @expansion{} @code{module}
@item
@  19. @tab @code{int}  @tab @expansion{} @code{number}
@item
@  20. @tab @code{int}  @tab @expansion{} @code{poly}
@item
@  21. @tab @code{number}  @tab @expansion{} @code{poly}
@item
@  22. @tab @code{string}  @tab @expansion{} @code{proc}
@item
@  23. @tab @code{list}  @tab @expansion{} @code{resolution}
@item
@  24. @tab @code{int}  @tab @expansion{} @code{vector}
  (i @expansion{} i*@code{gen(1)})
@item
@  25. @tab @code{poly}  @tab @expansion{} @code{vector}
  (p @expansion{} p*@code{gen(1)})
@end multitable

@subsubheading Type casting
@cindex Type casting
An expression can be casted to another type by using a type cast
expression:
@* type @code{(} expression @code{)}.

Possible type casts are:

@c ------------------------------------------------
@c Do not delete the leading `@ ' for indentation of the whole table.  The
@c table is indented since regular @tables are indented by a small amount,
@c too.
@c ------------------------------------------------
@multitable @columnfractions .05 .15 .83
@item
@    @tab to @tab from
@item
@    @tab  @code{ideal} @tab  expression lists of @code{int}, @code{number}, @code{poly}
@item
@    @tab  @code{ideal} @tab  @code{int}, @code{matrix}, @code{module}, @code{number}, @code{poly}, @code{vector}
@item
@    @tab  @code{int} @tab    @code{number}, @code{poly}
@item
@    @tab  @code{intvec} @tab expression lists of @code{int}, @code{intmat}
@item
@    @tab  @code{intmat} @tab @code{intvec} (@pxref{intmat type cast})
@item
@    @tab  @code{list} @tab   expression lists of any type
@item
@    @tab  @code{matrix} @tab @code{module}, @code{ideal},
@code{vector}, @code{matrix}.
@* There are two forms to convert something to a matrix: if @code{matrix(}
expression @code{)} is used then the size of the matrix is determined
by the size of expression.
@* But @code{matrix(} expression @code{,} m @code{,} n @code{)} may also be
used - the result is a
@tex
$ m \times n $
@end tex
@ifinfo
m x n
@end ifinfo
matrix (@pxref{matrix type cast})
@item
@    @tab  @code{module} @tab expression lists of @code{int}, @code{number},
@code{poly}, @code{vector}
@item
@    @tab  @code{module} @tab @code{ideal}, @code{matrix}, @code{vector}
@item
@    @tab  @code{number} @tab @code{poly}
@item
@   @tab  @code{poly} @tab   @code{int}, @code{number}
@item
@   @tab  @code{string} @tab   any type (@pxref{string type cast})
@end multitable

@strong{Example:}
@smallexample
@c computed example Type_casting general.doc:2743 
  ring r=0,x,(c,dp);
  number(3x);
@expansion{} 0
  number(poly(3));
@expansion{} 3
  ideal i=1,2,3,4,5,6;
  print(matrix(i));
@expansion{} 1,2,3,4,5,6
  print(matrix(i,3,2));
@expansion{} 1,2,
@expansion{} 3,4,
@expansion{} 5,6 
  vector v=[1,2];
  print(matrix(v));
@expansion{} 1,
@expansion{} 2 
  module(matrix(i,3,2));
@expansion{} _[1]=[1,3,5]
@expansion{} _[2]=[2,4,6]
@c end example Type_casting general.doc:2743
@end smallexample


@c ------------------------------------------------------------
@node Flow control, ,Type conversion and casting, The SINGULAR language
@subsection Flow control
@cindex Flow control
@cindex block

A block is a sequence of commands surrounded by @{ and @}.
@smallexample
@{
   command;
   @dots{}
@}
@end smallexample
Blocks are used whenever @sc{Singular} is used as a structured programming
language. The @code{if} and @code{else} structures allow conditional
execution of blocks (see @ref{if}, @ref{else}). @code{for} and @code{while}
loops are available for repeated execution of blocks (see @ref{for},
@ref{while}).
In procedure definitions the main part and the example section are blocks
as well(see @ref{proc}).
@menu
* break::
* quit::
* keepring::
* export::
* if::
* else::
* for::
* while::
* proc::
* return::
* continue::
@end menu


@c ----------------------------------------------------------------------
@c ----------------------------------------------------------------------
@node Input and output, Procedures, The SINGULAR language, General concepts
@section Input and output
@cindex input
@cindex output

@sc{Singular}'s input and output (short, I/O) is realized using links.
Links are the communication channels of @sc{Singular}, i.e., something
@sc{Singular} can write to and read from.  In this section, a short
overview of the usage of links and of the different link types is given.

For loading of libraries, see @ref{LIB}. For executing program scripts,
see @ref{filecmd}.

@subheading Monitoring
A special form of I/O is monitoring.  When monitoring is enabled,
@sc{Singular} makes a typescript of everything printed on your terminal
to a file.  This is
useful to create a protocol of a @sc{Singular} session.  The
@code{monitor} command enables and disables this feature
(@pxref{monitor}).


@subheading How to use links
Recall that links are the communication channels of @sc{Singular}, i.e.,
something @sc{Singular} can write to and read from using the functions
@code{write} and @code{read}. There are furthermore
the functions @code{dump} and @code{getdump} which store resp.@:
retrieve the content of an entire @sc{Singular} session to, resp.@: from,
a link.  The @code{dump} and @code{getdump} commands are not available
for DBM links.

For more information, see @ref{write}, @ref{read}, @ref{dump},
@ref{getdump}.

@strong{Example:}
@smallexample
@c computed example How_to_use_links general.doc:2830 
  ring r; poly p = x+y;
  dump("MPfile:w test.mp");   // dump the session to the file test.mp
  kill r;                     // kill the basering
  listvar();
@expansion{} // LIB                  [0]  string standard.lib
  getdump("MPfile:r test.mp");// read the dump from the file
  listvar();
@expansion{} // r                    [0]  *ring
@expansion{} //      p                    [0]  poly
@expansion{} // LIB                  [0]  string standard.lib
@c end example How_to_use_links general.doc:2830
@end smallexample

Specifying a link can be as easy as specifying a filename as a string.
Except for MPtcp links, links even do not need to be explicitly opened
or closed before, resp.@: after, they are used.  To explicitly open or
close a link, the @code{open}, resp.@: @code{close}, commands may be
used (see @ref{open}, @ref{close}).

Links have various properties which can be queried using the
@code{status} function (@pxref{status}).

@strong{Example:}
@smallexample
@c computed example How_to_use_links_1 general.doc:2851 
  link l = "MPtcp:fork";
  l;
@expansion{} // type : MPtcp
@expansion{} // mode : fork
@expansion{} // name : 
@expansion{} // open : no
@expansion{} // read : not ready
@expansion{} // write: not ready
  open(l);
  status(l, "open");
@expansion{} yes
  close(l);
  status(l, "open");
@expansion{} no
@c end example How_to_use_links_1 general.doc:2851
@end smallexample

@subheading ASCII links
Data that can be converted to a string that can be written into files for
storage or communication with other programs. The data are written in
plain ASCII format. Reading from an ASCII link returns a string ---
conversion into other data is up to the user. This can be done, for
example, using the command @code{execute} (@pxref{execute}).

ASCII links should primarily be used for storing small amounts of data,
especially if it might become necessary to manually inspect or
manipulate the data.

@xref{ASCII links}, for more information.

@strong{Example:}
@smallexample
@c computed example ASCII_links general.doc:2876 
  // (over)write file test.ascii, link is specified as string
  write(":w test.ascii", "int i =", 3, ";");
  // reading simply returns the string
  read("test.ascii");
@expansion{} int i =
@expansion{} 3
@expansion{} ;
@expansion{} 
  // but now test.ascii is "executed"
  execute(read("test.ascii"));
  i;
@expansion{} 3
@c end example ASCII_links general.doc:2876
@end smallexample

@subheading MPfile links
Data is stored in the binary MP format.  Read and write access is very
fast compared to ASCII links.  All data (including such data that cannot
be converted to a string) can be written to an MPfile link.  Reading
from an MPfile link returns the written expressions (i.e., not a string,
in general).

MPfile links should primarily be used for storing large amounts of data
(like dumps of the content of an entire @sc{Singular} session), and if
the data to be stored cannot be easily converted from or to a string
(like rings, or maps).

MPfile links are implemented on Unix-like operating systems only.

@xref{MPfile links}, for more information.

@strong{Example:}
@smallexample
@c computed example MPfile_links general.doc:2905 
  ring r;
  // (over)write MPfile test.mp, link is specified as string
  write("MPfile:w test.mp", x+y);
  kill r;
  def p = read("MPfile:r test.mp");
  typeof(p); p;
@expansion{} poly
@expansion{} x+y
@c end example MPfile_links general.doc:2905
@end smallexample

@subheading MPtcp links
Data is communicated with other processes (e.g., @sc{Singular}
processes) which may run on the same computer or on different ones.  Data
exchange is accomplished using TCP/IP links in the binary MP format.
Reading from an MPtcp link returns the written expressions (i.e., not a
string, in general).

MPtcp links should primarily be used for communications with other
programs or for parallel computations (see, for example,
@ref{Parallelization with MPtcp links}).

MPtcp links are implemented on Unix-like operating systems only.

@xref{MPtcp links}, for more information.

@strong{Example:}
@smallexample
@c computed example MPtcp_links general.doc:2932 
  ring r;
  link l = "MPtcp:launch"; // declare a link explicitly
  open(l);  // needs an open, launches another SINGULAR as a server
  write(l, x+y);
  kill r;
  def p = read(l);
  typeof(p); p;
@expansion{} poly
@expansion{} x+y
  close(l); // shuts down SINGULAR server
@c end example MPtcp_links general.doc:2932
@end smallexample

@subheading DBM links
Data is stored in and accessed from a data base.  Writing is
accomplished by a key and a value and associates the value with the key
in the specified data base.  Reading is accomplished w.r.t.@: a key,
the value associated to it is returned.  Both the key and the value have to
be specified as strings.  Hence, DBM links may be used only for data
which may be converted to or from strings.

DBM links should primarily be used when data needs to be accessed not in a
sequential way (like with files) but in an associative way (like with
data bases).

@xref{DBM links}, for more information.

@strong{Example:}
@smallexample
@c computed example DBM_links general.doc:2960 
  ring r;
  // associate "x+y" with "mykey"
  write("DBM:w test.dbm", "mykey", string(x+y));
  // get from data base what is stored under "mykey"
  execute(read("DBM: test.dbm", "mykey"));
@expansion{} x+y
@c end example DBM_links general.doc:2960
@end smallexample


@c ---------------------------------------------------------------------------
@node Procedures, Libraries, Input and output, General concepts
@section Procedures
@cindex Procedures

Procedures contain sequences of commands of the @sc{Singular} language.
They are used to extend
the set of commands with user defined commands.  Procedures are defined
by either typing them on the command line or by loading them from a
so-called library file with the LIB command, see @ref{LIB}.  Procedures are
invoked like normal built-in commands, i.e., by typing their name
followed by the list of arguments in parentheses. The invocation then
executes the sequence of commands stored in the specified procedure. All
defined procedures can be displayed by the command @code{listvar(proc);}.

@menu
* Procedure definition::
* Names in procedures::
* Parameter list::
* Procedure commands::
@end menu
@c -------------------------------------------------
@node Procedure definition, Names in procedures, Procedures, Procedures
@subsection Procedure definition
@cindex Procedure definition
@cindex procedures, static
@cindex static procedures
@c ------------------------------------------------------------
@c   This piece of text exists also in the file types.doc,
@c   chapter "proc declaration".
@c   If you change something here, change it there, too!
@c ------------------------------------------------------------
@table @strong
@item Syntax:
[@code{static}] @code{proc} proc_name [parameter_list]
@*["help_text"]
@*@code{@{}
@*
@tex
\quad
@end tex
   procedure_body
@*@code{@}}
@*[@code{example}
@*@code{@{}
@*
@tex
\quad
@end tex
   sequence_of_commands;
@*@code{@}}]
@item Purpose:
defines a new function, the @code{proc} proc_name, with the additional
information help_text, which is copied to the screen by
@code{help proc_name;} and the @code{example} section
which is executed by @code{example proc_name;}.
@* The help_text, the parameter_list, and the example section are optional.
The default for a parameter_list is @code{(list #)}, see @ref{Parameter list}.
The help and example sections are ignored if the procedure is defined
@ifset namespaces
interactively, i.e., if it was not loaded from a file by a @ref{load} command.
@end ifset
@ifclear namespaces
interactively, i.e., if it was not loaded from a file by a LIB command
(@ref{LIB}).
@end ifclear
@* Specifying @code{static} in front of the proc-definition (in a library file)
makes this procedure local to the library, i.e., accessible only for the other
procedures in the same library, but not for the users. So there is no
reason anymore to define a procedure within another one (it just makes
debugging harder).
@end table

@subsubheading Example of an interactive procedure definition
@smallexample
@c computed example Example_of_an_interactive_procedure_definition general.doc:3045 
  proc milnor_number (poly p)
  @{
    ideal i= std(jacob(p));
    int m_nr=vdim(i);
    if (m_nr<0)
    @{
      "// not an isolated singularity";
    @}
    return(m_nr);         // the value of m_nr is returned
  @}
  ring r1=0,(x,y,z),ds;
  poly p=x^2+y^2+z^5;
  milnor_number(p);
@expansion{} 4
@c end example Example_of_an_interactive_procedure_definition general.doc:3045
@end smallexample

@subsubheading Example of a procedure definition in a library
First, the library definition:
@smallexample
@c begin included file sample.lib from general.doc:3065
// Example of a user accessible procedure
proc tab (int n)
"USAGE:    tab(n);  (n integer)
RETURNS:  string of n space tabs
EXAMPLE:  example tab; shows an example"
@{ return(internal_tab(n)); @}
example
@{
  "EXAMPLE:"; echo=2;
  for(int n=0; n<=4; n=n+1)
  @{ tab(4-n)+"*"+tab(n)+"+"+tab(n)+"*"; @}
@}

// Example of a static procedure
static proc internal_tab (int n)
@{ return(" "[1,n]); @}
@c end included file from general.doc:3065
@end smallexample

Now, we load the library and execute the procedures defined there:
@smallexample
@c computed example Example_of_a_procedure_definition_in_a_library general.doc:3070 
  LIB "sample.lib";        // load the library sample.lib
  example tab;             // show an example
@expansion{} // proc tab from lib sample.lib
@expansion{} EXAMPLE:
@expansion{}   for(int n=0; n<=4; n=n+1)
@expansion{}   @{ tab(4-n)+"*"+tab(n)+"+"+tab(n)+"*"; @}
@expansion{}     *+*
@expansion{}    * + *
@expansion{}   *  +  *
@expansion{}  *   +   *
@expansion{} *    +    *
@expansion{} 
  "*"+tab(3)+"*";          // use the procedure tab
@expansion{} *   *
  // the static procedure internal_tab is not accessible
  "*"+internal_tab(3)+"*";
@expansion{}    ? 'sample.lib::internal_tab()' is a local procedure and cannot be acce\
   ssed by an user.
@expansion{}    ? error occurred in line 5: `  "*"+internal_tab(3)+"*";`
  // show the help section for tab
  help tab;
@expansion{} // ** Could not get IdxFile. 
@expansion{} // ** Either set environment variable SINGULAR_IDX_FILE to IdxFile,
@expansion{} // ** or make sure that IdxFile is at /home/hannes/singular/2-0/doc/singu\
   lar.idx
@expansion{} // proc tab from lib sample.lib
@expansion{} proc tab (int n)
@expansion{} USAGE:    tab(n);  (n integer)
@expansion{} RETURNS:  string of n space tabs
@expansion{} EXAMPLE:  example tab; shows an example
@c end example Example_of_a_procedure_definition_in_a_library general.doc:3070
@end smallexample

@subsubheading Guidelines for the help text of a procedure
There are no enforced rules on the format of the help section of a
procedure.

Nevertheless, we recommend that the help text of a procedure should
contain information about the usage,
purpose, return values and generated objects. Particular assumptions or
limitations should be listed. It should also be mentioned if global
objects are generated or manipulated.

The help text of procedures contained in libraries of the
@sc{Singular} distribution
should furthermore comply with  certain rules as explained in
@ref{The help string of procedures}.

@c ---------------------------------------
@node Names in procedures, Parameter list, Procedure definition, Procedures
@subsection Names in procedures
@cindex Names in procedures
@cindex local names

All variables are local to the procedure they are defined in.
Locally defined variables cannot interfere with names in other
procedures and are automatically deleted after leaving
the procedure.

Internally, local variables are stored using the nesting level. A variable
is said to have nesting level 1, if it is local to a procedure that was
called interactively, nesting level 2, if it is local to a procedure that
was called by a procedure of nesting level 1 etc. @code{listvar()} also
displays the nesting level, nesting level 0 is used for global objects
(see @ref{listvar}). A ring may be 'moved up' by one nesting level
with @code{keepring} (see @ref{keepring}). All variables living in that
ring are moved together with that ring.

To keep local variables after leaving the procedure, they have to be
exported (i.e. made known) to some higher level by the @code{export}
command (@pxref{export}).

@strong{Example:}
@smallexample
@c computed example Names_in_procedures general.doc:3122 
proc xxx
@{
  int k=4;        //defines a local variable k
  int result=k+2;
  export(result);  //defines the global variable "result".
@}
xxx();
listvar(all);
@expansion{} // result               [0]  int 6
@expansion{} // LIB                  [0]  string standard.lib
@c end example Names_in_procedures general.doc:3122
@end smallexample
Note that the variable @code{result} became a global variable after the
execution of @code{xxx}.

@c ---------------------------------------
@node Parameter list, Procedure commands, Names in procedures, Procedures
@subsection Parameter list
@cindex Parameter list
@table @code
@item @strong{Syntax:}
@code{( )}
@*@code{(} parameter_definition @code{)}
@item @strong{Purpose:}
defines the number, type and names of the
arguments to a @code{proc}.
@*The parameter_list is optional.
The default for a parameter_list is @code{(list #)}
which means the arguments are referenced by @code{#[1], #[2]}, etc.
@item @strong{Example:}
@smallexample
proc x0
@{
    // can be called with
... // any number of arguments of any type: #[1], #[2],...
    // number of arguments: size(#)
@}

proc x1 ()
@{
... // can only be called without arguments
@}

proc x2 (ideal i, int j)
@{
... // can only be called with 2 arguments,
    // which can be converted to ideal resp. int
@}

proc x3 (i,j)
@{
... // can only be called with 2 arguments
    // of any type
    // (i,j) is the same as (def i,def j)
@}

proc x5 (i,list #)
@{
... // can only be called with at least 1 argument
    // number of arguments: size(#)+1
@}

@end smallexample
@item @strong{Note:}
@* The parameter_list may stretch across multiple lines.
@* A parameter may have any type (including the types @code{proc}
   and @code{ring}). If a parameter is of type ring, then it
   can only be specified by name, but not with a type, e.g.
@smallexample
proc x6 (r)
@{
... // this is correct even if the parameter is a ring
@}

proc x7 (ring r)
@{
... // this is NOT CORRECT
@}
@end smallexample
@end table

@c --------------------------------------- ----------
@node Procedure commands, , Parameter list, Procedures
@subsection Procedure commands
@cindex Procedure commands

Some commands only make sense inside a procedure, since they make
objects known to the nesting level from which the procedure was called
or to all nesting levels.

@c inserted refs from general.doc:3211
@ifinfo
@menu
See
* export::
* keepring::
* return::
@end menu
@end ifinfo
@iftex
See
@ref{export};
@ref{keepring};
@ref{return}.
@end iftex
@c end inserted refs from general.doc:3211

@c ----------------------------------------------------------------------
@node Libraries, Guidelines for writing a library, Procedures, General concepts
@section Libraries
@cindex Libraries

A library is a collection of @sc{Singular}
procedures in a file.
@*@sc{Singular} reads a library with the command
@code{LIB}. General information about the library is displayed by the command
@code{help libname_lib}. After loading the library, its procedures can be used
like the built-in @sc{Singular} functions.

To have the full functionality of a built-in function libraries have to
comply with the few syntax rules described below.

Furthermore, libraries which are to be included in the @sc{Singular}
distribution, have to comply with certain rules as explained in
@ref{Guidelines for writing a library}.

@menu
* Loading of a library::
* Format of a library::
@end menu

@c ---------------------------------------
@node Loading of a library, Format of a library,, Libraries
@subsection Loading of a library
@cindex Loading of a library
@cindex path
@cindex SearchPath
@cindex RootDir
@cindex BinDir
@cindex DefaultDir
@cindex SINGULARPATH

Libraries can be loaded with the @code{LIB} command:

@c ------------------------------------------------------------
@c   This piece of text exists also in the file reference.doc,
@c   chapter "LIB".
@c   chapter "load".
@c   If you change something here, change it there, too!
@c ------------------------------------------------------------
@table @code
@item @strong{Syntax:}
@code{LIB} string_expression @code{;}
@item @strong{Type:}
none
@item @strong{Purpose:}
reads a library of procedures from a file. If the given filename does
not start with @kbd{.} or @kbd{/} and can not be located in the current
directory, each directory contained in the library
@code{SearchPath} is searched for file of this name.
@item @strong{Note on SearchPath:}
The the library @code{SearchPath} is constructed at @sc{Singular}
start-up time as follows:
@enumerate
@item
the directories contained in the environment variable
@code{SINGULARPATH} are appended

@item
the directories @code{$BinDir/LIB}, @code{$RootDir/LIB},
@code{$RootDir/../LIB}, @code{$DefaultDir/LIB}, @code{$DefaultDir/../LIB}
are appended, where
@itemize @bullet
@item @code{$BinDir}
is the value of the environment variable
@code{SINGULAR_BIN_DIR}, if set, or, if not set, the directory in which the
@sc{Singular} program
resides
@item @code{$RootDir}
is the value of the environment variable
@code{SINGULAR_ROOT_DIR}, if set, or, if not set, @code{$BinDir/../}.
@item @code{$DefaultDir}
is the value of the environment variable
@code{SINGULAR_DEFAULT_DIR}, if set, or @code{/usr/local/Singular/} on a
Unix platform, @code{\Singular\} on a Windows 95/98/NT platform, and
@code{::Singular:} on a Macintosh.
@end itemize

@item
all directories which do not exist are removed from the @code{SearchPath}.
@end enumerate

For setting environment variables see @ref{system}, or consult the manual of
your shell.

The library @code{SearchPath} can be examined  by starting up
@sc{Singular} with the option @code{-v}, or by issuing the command
@code{system("--version");}.

@item @strong{Note on standard.lib:}
Unless  @sc{Singular} is started with the @code{--no-stdlib} option, the
library @code{standard.lib} is automatically loaded at start-up time.

@end table

Only the names of the procedures in the library are loaded, the body of
the procedures is read during the first call of this procedure. This minimizes
memory consumption by unused procedures.  When @sc{Singular} is started
with the @code{-q} or @code{--quiet} option, no message about the
loading of a library is displayed.  More precisely, option @code{-q} (and
likewise @code{--quiet}) unsets option @code{loadLib} to inhibit
monitoring of library loading (@pxref{option}).

All loaded libraries are displayed by the @code{LIB;}
command:
@smallexample
@c computed example Loading_of_a_library general.doc:3327 
  option(loadLib);   // show loading of libraries;
                     // standard.lib is loaded
  LIB;
@expansion{} standard.lib
                     // the names of the procedures of inout.lib
  LIB "inout.lib";   // are now known to Singular
@expansion{} // ** loaded inout.lib (1.21.2.5,2002/06/12)
  LIB;
@expansion{} standard.lib,inout.lib
@c end example Loading_of_a_library general.doc:3327
@end smallexample

@c inserted refs from general.doc:3337
@ifinfo
@menu
See
* Command line options::
* LIB::
* Procedures and libraries::
* SINGULAR libraries::
* proc::
* standard_lib::
* string::
* system::
@end menu
@end ifinfo
@iftex
See
@ref{Command line options};
@ref{LIB};
@ref{Procedures and libraries};
@ref{SINGULAR libraries};
@ref{proc};
@ref{standard_lib};
@ref{string};
@ref{system}.
@end iftex
@c end inserted refs from general.doc:3337

@c ---------------------------------------
@node Format of a library,, Loading of a library, Libraries
@subsection Format of a library
@cindex Format of a library

A library file can contain comments, a category-, info- and version-string
definition, @code{LIB} commands, @code{proc} commands
and @code{proc} commands with @code{example} and @code{help} sections,
i.e., the following keywords are allowed:
@code{category},
@code{info},
@code{version},
@code{LIB},
@code{/* ... */},
@code{//},
@code{[static] proc}.
Anything else is not recognized by the parser of @sc{Singular} and leads
to an error message while loading the library.
If an error occurs, loading is aborted and an error message is displayed,
specifying the type of error and the line where it was detected.

The category-, info- and version-string are defined as follows:
@cindex  info in a library
@table @code
@item @strong{Syntax:}
@code{info} = string_constant @code{;}
@item @strong{Purpose:}
defines the general help for the library. This text is displayed on
@code{help libname_lib;}.
@item @strong{Example:}
@smallexample
info="
      This could be the general help of a library.
     Quotes must be escaped with a \ such as \"
";
@end smallexample
@item @strong{Note:}
In the info-string the characters \ and " must be preceded by a \ (escaped).
It is recommended that the info string is placed on the top of a
library file and contains general information about the library as well
as a listing of all procedures available to the users (with a one line
description of each procedure).

Although there is no enforced format of the info string of a library, we
recommend that you follow certain rules as explained in
@ref{The help string of a library}.
@end table

@cindex  version in a library
@table @code
@item @strong{Syntax:}
@code{version} = string_constant @code{;}
@item @strong{Purpose:}
defines the version number for the library. It is displayed when the
library is loaded.
@item @strong{Example:}
@smallexample
version="$ I@i{}@comment
@comment !! Do not erase the @comment at the end of the previous line !!
@comment !! It prevents cvs to substitute the id string                !!
d: sample.lib,v 1.2 1998/05/07 singular Exp $";
version="some version string";
@end smallexample
@item @strong{Note:}
It is common practice to simply define the version string to be
@code{"$ I}@comment
@comment !! Do not erase the @comment at the end of the previous line !!
@comment !! It prevents cvs to substitute the id string                !!
@code{d:$"} and let a version control system expand it.
@end table

@cindex  category in a library
@table @code
@item @strong{Syntax:}
@code{category} = string_constant @code{;}
@item @strong{Purpose:}
defines the category for the library.
@item @strong{Example:}
@smallexample
category="Utilities";
@end smallexample
@item @strong{Note:}
reserved for sorting the libraries into categories.
@end table

@c ---------------------------------------
@node Guidelines for writing a library, Debugging tools, Libraries, General concepts
@section Guidelines for writing a library
@cindex Guidelines for writing a library
@cindex library, Guidelines for writing

Although there are very few enforced rules on how libraries are written
(see @ref{Libraries}), it is recommended that the libraries comply with
the guidelines explained in this section, so that debugging and
understanding are made easier.

@strong{Note:} For libraries which are to be included in the @sc{Singular}
distribution, the following guidelines are mandatory.

@menu
* Procedures in a library::
* Documentation of a library::
* Typesetting of help strings::
* The help string of a library::
* The help string of procedures::
* template_lib::
@end menu

@c ---------------------------------------
@node Procedures in a library, Documentation of a library,,Guidelines for writing a library

@subsection Procedures in a library
@cindex Procedures in a library
@cindex library, procedures

In this section we list miscellaneous recommendations on how procedures
contained in a library should be implemented.
@enumerate
@item
The info- and version-string should appear at the beginning of the
library, before procedure definitions.

@item
The info-string should have the format as explained in @ref{The help
string of a library}.

@item
Each procedure which should not be accessible by users should be
declared @code{static}.

@item
Each procedure which is not declared @code{static} should have a
help and example section as explained in @ref{Procedure definition}.
@*Such procedures should furthermore carefully check any assumptions
made about their input (like the type of list elements), and, if
necessary, omit an error using the  function @ref{ERROR}.

@item
No procedures should be defined within the body of another procedure.

@item
If the value of
the reserved variable @code{printlevel} (@pxref{printlevel}) is greater
than 0 then interactive user-input, i.e., the usage of functions like
@code{pause("..")} or @code{read("");} (@pxref{read}), may be requested.

@item
If the value of
the reserved variable @code{printlevel} (@pxref{printlevel}) is 0
then interactive user-input, i.e., the usage of functions like
@code{pause("..")} or @code{read("");} (@pxref{read}), may @strong{not} be
requested. Instead, an error (using the function
@ref{ERROR}) should be reported together with the recommendation on
increasing the value of the reserved variable @code{printlevel}.

@item
It is often useful for a procedure to print out comments, either to
explain results or to display intermediate computations. However,
if this procedure is called by another procedure, such comments are
confusing and disturbing in most cases.

@sc{Singular} offers an elegant solution, which requires the usage of
the @sc{Singular} function @ref{dbprint} and the reserved variables
@ref{printlevel}, and @ref{voice} (@code{voice} counts the nesting of
procedures; It has the value 1 on the top level, 2 inside the first
procedure etc.; @code{printlevel} has the value 0 by default, but can be
set to any integer value by the user).

For example, if the following procedure @code{Test} is called
directly from the top level then @samp{comment1} is displayed (i.e.,
printed out) but not
@samp{comment2}; and nothing is displayed if  @code{Test} is called
from within any other procedure.
However, if @code{printlevel} is set to  a value k with k>0, then
@samp{comment1} (resp.@: @samp{comment2}) is displayed provided that
@code{Test} is called from other procedures, with a nesting level up to
k (resp.@: k-1).

Note furthermore, that the example part of a procedure
behaves in this respect like a procedure (i.e., the value of @code{voice}
is 1). Therefore, the command @code{printlevel=1;} is necessary for
@samp{comment1} to be displayed on @code{ example Test;}. However,
since printlevel is a global variable, it should be reset to the old
value at the end of the example part.

@smallexample
proc Test
"USAGE:   ...
         ...
EXAMPLE: example Test; shows an example
"
@{   ...
   int p = printlevel - voice + 3;
    ...
   dbprint(p,"comment1");
   dbprint(p-1,"comment2");
   // dbprint prints only if p > 0
    ...
@}
example
@{ "EXAMPLE:"; echo = 2;
   int p = printlevel;   //store old value of printlevel
   printlevel = 1;       //assign new value to printlevel
    ...
   Test();
   printlevel = p;       //reset printlevel to old value
@}
@end smallexample

@end enumerate
@c @end table

@c ---------------------------------------
@node Documentation of a library, Typesetting of help strings, Procedures in a library, Guidelines for writing a library

@subsection Documentation of a library
@cindex Documentation of a library
@cindex library, documentation

The typesetting language in which the @sc{Singular} documentation is
written is @code{texinfo}. Based on various tools, @code{info, dvi, ps,} and
@code{html} versions of the @code{texinfo} documentation are generated.


Starting with @sc{Singular} version 1-3, the @code{texinfo}
documentation of all libraries of the @sc{Singular} distribution is
generated automatically from their source code.

More precisely, for each library,
@itemize @minus
@item
the info string of the library is parsed and typeset
as explained in @ref{Typesetting of help strings}.

@item
the help string of each procedure listed in the @code{PROCEDURE:}
section of the library info string is parsed
and typeset as explained in @ref{Typesetting of help strings}.

@item
the example of each procedure listed in the @code{PROCEDURE:} section of
the library info string is computed and its output is
included into the documentation.
@end itemize

For a uniform look-and-feel of the library documentation, library
developers should
@itemize @minus
@item
follow the recommendation of
@ref{The help string of a library} and @ref{The help string of
procedures}.

@item
consult the source code of libraries like
@code{template.lib} (@pxref{template_lib})
for examples on how library documentations are
written.

@item
make sure that each procedure listed in the @code{PROCEDURE:}
section of the library info string has a help string and an example
section.

@item
not use interactive functions like
@code{pause("..")} or @code{read("");} (@pxref{read}) and should limit the
length of input lines to 60 characters in the example section of
procedures.

@item
carefully check the generated
documentation of their libraries in its various formats using the
@code{lib2doc} (@pxref{lib2doc}) utility.
@end itemize

@menu
* lib2doc::
@end menu

@node lib2doc,,,Documentation of a library
@subsubsection lib2doc
@cindex lib2doc

@include lib2doc.texi

@c ----------------------------------------
@node Typesetting of help strings, The help string of a library, Documentation of a library, Guidelines for writing a library
@subsection Typesetting of help strings
@cindex Typesetting of help strings
@cindex library, typesetting of help strings

The help strings of procedures and info strings of libraries which are
included in the
distribution of @sc{Singular} are parsed and automatically converted
into the texinfo format (the typesetting language in which the
documentation  of @sc{Singular} is written).

For optimal typesetting results, the guidelines for writing libraries
and procedures should be followed, and the following points should be
kept in mind:

@itemize @bullet
@item
If a help string starts with an @code{@@} sign, then no parsing is done,
and the help string is assumed to be already in the texinfo format.
@item
help strings are typeset within a @code{@@table @@asis} environment
(which is similar to a latex @code{description} environment).
@item
If a line starts with  only uppercase words and contains a colon, then
the text up
to the colon is taken to be the description-string of an item and the
text following the colon is taken to be the content of the item.
@item
If the description-string of an item matches
@table @asis
@item @code{EXAMPLE}
then this item and its content is ignored.
@item @code{SEE ALSO}
then the
content of the item is assumed to be comma-separated words which are
valid references to other texinfo nodes of the manual. (e.g., all
procedure and command names are also texinfo nodes).
@item  @code{KEYWORDS} (or, @code{KEYPHRASES})
then the content of the item is assumed to be a
semicolon-separated list of phrases which are taken as keys for the
index of the manual (N.B. the name of a procedure/library is
automatically added to the index keys).
@item @code{PROCEDURES}
then the
content of the item is assumed to be a summary description of the
procedures contained in the library. Separate texinfo nodes (subsections
in printed documents) are
@strong{only} created out of the help strings of such procedures which
appear in the summary description of a library.
@item @code{LIBRARY}
then the content of the item is assumed to be a one-line description of
a library. If this one-line description consist of only uppercase
characters, then it is typeset in all lowercase characters in the
manual (otherwise it is left as is).
@end table

@item
For the content of an item, the following texinfo markup elements
are recognized (and, their content not further manipulated):
@table @asis
@item @code{@@*}
to enforce a line-break.
@table @asis
@item Example:
@code{old line @@* new line}
@* @expansion{}
@* old line @* new line
@end table

@item @code{@@ref@{...@}}
References to other parts of the @sc{Singular} manual can be set using
one of the following @code{@@ref@{node@}} constructs. Notice that
@code{node} must be the name of a section of the @sc{Singular}
manual. In particular, it may be a name of a function, library or library
procedure.

@table @asis
@item @code{@@xref@{node@}}
for a reference to the node @code{node} at the beginning of a sentence.
@item @code{@@ref@{node@}}
for a reference to the node @code{node} at the end of a sentence.
@item @code{@@pxref@{node@}}
for a reference to the node @code{node} within parenthesis.

@item Example:
@code{@@xref@{Tropical Storms@}, for more info.}
@*@expansion{}*Note Hurricanes::, for more info.
@*@expansion{}See Section 3.1 [Hurricanes], page 24, for more info.

@code{For more information, see @@ref@{Hurricanes@}.}
@*@expansion{}For more information, see *Note Hurricanes::.
@*@expansion{}For more information, see Section 3.1 [Hurricanes], page 24.


@code{... storms cause flooding (@@pxref@{Hurricanes@}) ...}
@*@expansion{}... storms cause flooding (*Note Hurricanes::) ...
@*@expansion{}... storms cause flooding (see Section 3.1 [Hurricanes],
page 24)
@end table

@item @code{@@math@{..@}}
for typesetting of small (i.e., which do not go over
multiple lines) mathematical expressions  in LaTeX math-mode
syntax.
@table @asis
@item Example:
@code{@@math@{\alpha@}}
@*@expansion{}
@*
@ifinfo
@math{\alpha}
@end ifinfo
@tex
$\alpha$
@end tex

@item Note:
Mathematical expressions inside @code{@@math@{..@}} may
not contain curly parenthesis and the "at" sign, i.e., may not contain
@code{@{,@},@@}.
@end table

@item @code{@@code@{..@}}
for typesetting of small (i.e., which do not go over
multiple lines) strings in typewriter font.
@table @asis
@item Example:
@code{@@code@{typewriter font@}}
@*@expansion{}
@*@code{typewriter font}
@item Note:
The string inside @code{@@code@{..@}} may
not contain curly parenthesis and the "at" sign, i.e., may not contain
@code{@{,@},@@}.
@end table

@item @code{@@example}
 @code{...}
@item @code{@@end example}
for pre-formatted text which is indented and typeset in typewriter
font.
@table @asis
@item Example:
@example
before example
@@example
in              example
notice extra identation and
escape of special characters like @@@{,@@@},@@@@
@@end example
after example
@end example
@expansion{}
@*before example
@example
in                example
notice extra identation  and
escape of special characters like @{,@},@@
@end example
after example
@item Note:
The characters @{,@},@@ have to be escaped by an @@ sign inside an
@@example environment.
@end table

@item @code{@@format}
 @code{...}
@item @code{@@end format}
for pre-formatted text which is not indented and typeset in normal
font.
@table @asis
@item Example:
@example
before format
@@format
in              format
no extra identation but still
escape of special characters like @@@{,@@@},@@@@
@@end format
after format
@end example
@expansion{}
@*before format
@format
in              format
no extra identation  but still
escape of special characters like @{,@},@@
@end format
after format
@item Note:
The characters @{,@},@@ have to be escaped by an @@ sign inside an
@@example environment.
@end table


@item @code{@@texinfo}
 @code{...}
@item @code{@@end texinfo}
for text which is written in pure texinfo.
@table @asis
@item Example:
@example
@@texinfo
Among others, within a texinfo environment
one can use the tex environment to typeset
more complex mathematical like
@@tex
$i_@{1,1@} $
@@tex
@@end texinfo
@end example
@*@expansion{}
@*Among others, within a texinfo environment one can use the tex environment
to typeset more complex mathematical like
@tex
$ i_{1,1} $
@end tex
@end table

@end table

Furthermore, a line-break is inserted in front of each line
whose previous line is shorter than 60
characters and does not contain any of the above described recognized
texinfo markup elements.
@end itemize

See also @ref{template_lib} for an examples of the typesetting rules
explained here.

@c ----------------------------------------
@node The help string of a library, The help string of procedures, Typesetting of help strings, Guidelines for writing a library
@subsection The help string of a library
@cindex help string of a library
@cindex library, help string
@cindex library, info string
@cindex info string of a library

The help (or, info) string of a library should have the following format:

@smallexample
info="
LIBRARY: <library_name> <one line description of the content>
AUTHOR:  <name, and email address of author>
[SEE ALSO: <comma-separated words of cross references>]
[KEYWORDS: <semicolon-separated phrases of index keys>]
PROCEDURES:
  <procedure1>;     <one line description of the purpose>
   .
   .
  <procedureN>;     <one line description of the purpose>
";
@end smallexample
Only such procedures should be listed in the @code{PROCEDURE} section
which are not @code{static} and which
have a help and example section.

The purpose of the one line procedure descriptions is not to give a short help
for the procedure, but to help the user decide what procedure might be
the right one for the job. Details can then be found in the help section
of each procedure. Therefore parameters may be omitted or abbreviated if
necessary. If this description consists of only upper-case characters,
then it will be typeset in all lowercase characters in the manual.

For more information, see @ref{Typesetting of help strings}. For an
example, see @ref{template_lib}.


@c ----------------------------------------
@node The help string of procedures, template_lib, The help string of a library, Guidelines for writing a library
@subsection The help string of procedures
@cindex help string of a procedure
@cindex procedures, help string

The help string of a procedure should have the following format:

@smallexample
USAGE:    <proc_name>(<parameters>);   <explanation of parameters>
[CREATE:  <description of created objects which are not returned>]
RETURN:   <description of the purpose and return value>
[NOTE:    <particular assumptions or limitations, details>]
[SEE ALSO: <comma-separated names of related procedures/cross references>]
[KEYWORDS: <semicolon-separated phrases of index keys>]
EXAMPLE:  example <proc_name>; shows an example
@end smallexample

Further arbitrary items (like @code{THEORY:}, or @code{BACKGROUND:}) are
recognized, as well, but should be used diligently.

Remember that help strings are formatted as explained in
@ref{Typesetting of help strings}. In particular, descriptions may
contain the texinfo markup elements @code{@@*, @@math@{..@},
@@code@{..@}, @@example, @@format, @@texinfo} to better control their
typesetting. See @ref{msum}, @ref{mdouble}, @ref{mtripple} for examples.

@c ----------------------------------------
@node template_lib,, The help string of procedures, Guidelines for writing a library
@subsection template_lib
@cindex template_lib
@cindex Template for writing a library
@cindex library, template

First, we show the source-code of a template library:
@smallexample
@c begin included file template.lib from general.doc:3933
////////////////////////////////////////////////////////////////////
// version string automatically expanded by CVS

version="Id: general.tex,v 1.1 2003/08/08 14:27:06 pertusus Exp $";
category="Miscellaneous";
// summary description of the library
info="
LIBRARY:   template.lib  A Template for a Singular Library
AUTHOR:    Olaf Bachmann, email: obachman@@mathematik.uni-kl.de

SEE ALSO:  standard_lib, Guidelines for writing a library,
           Typesetting of help strings

KEYWORDS: library, template.lib; template.lib; library, info string

PROCEDURES:
  mdouble(int)           return double of int argument
  mtripple(int)          return three times int argument
  msum([int,..,int])     sum of int arguments
";
////////////////////////////////////////////////////////////////////
proc mdouble(int i)
"USAGE:    mdouble(i); i int
RETURN:   int: i+i
NOTE:     Help string is in pure ASCII
          this line starts on a new line since previous line is short
          mdouble(i): no new line
SEE ALSO: msum, mtripple, Typesetting of help strings
KEYWORDS: procedure, ASCII help
EXAMPLE:  example mdouble; shows an example"
@{
  return (i + i);
@}
example
@{ "EXAMPLE:"; echo = 2;
  mdouble(0);
  mdouble(-1);
@}
////////////////////////////////////////////////////////////////////
proc mtripple(int i)
"@@c we do texinfo here
@@table @@asis
@@item @@strong@{Usage:@}
@@code@{mtripple(i)@}; @@code@{i@} int

@@item @@strong@{Return:@}
int: @@math@{i+i+i@}
@@item @@strong@{Note:@}
Help is in pure Texinfo
@@*This help string is written in texinfo, which enables you to use,
among others, the @@@@math command for mathematical typesetting (like
@@math@{\alpha, \beta@}).
@@*It also gives more control over the layout, but is, admittingly,
more cumbersome to write.
@@end table
@@c use @@c ref contstuct for references
@@cindex procedure, texinfo help
@@c ref
@@strong@{See also:@}
@@ref@{mdouble@}, @@ref@{msum@}, @@ref@{Typesetting of help strings@}
@@c ref
"
@{
  return (i + i + i);
@}
example
@{ "EXAMPLE:"; echo = 2;
  mtripple(0);
  mtripple(-1);
@}
////////////////////////////////////////////////////////////////////
proc msum(list #)
"USAGE:  msum([i_1,..,i_n]); @@code@{i_1,..,i_n@} def
RETURN:  Sum of int arguments
NOTE:    This help string is written in a mixture of ASCII and texinfo
         @@* Use a @@ref constructs for references (like @@pxref@{mtripple@})
         @@* Use @@code  for typewriter font (like @@code@{i_1@})
         @@* Use @@math  for simple math mode typesetting (like @@math@{i_1@}).
         @@* Note: No parenthesis like @} are allowed inside @@math and @@code
         @@* Use @@example for indented preformatted text typeset in typewriter
         font like
@@example
         this  --> that
@@end example
        Use @@format  for preformatted text typeset in normal font
@@format
         this --> that
@@end format
        Use @@texinfo for text in pure texinfo
@@texinfo
@@expansion@{@}
@@tex
$i_@{1,1@}$
@@end tex

@@end texinfo
        Notice that
        automatic linebreaking         is still in affect (like on this line).
SEE ALSO: mdouble, mtripple, Typesetting of help strings
KEYWORDS: procedure, ASCII/Texinfo help
EXAMPLE: example msum; shows an example"
@{
  if (size(#) == 0) @{ return (0);@}
  if (size(#) == 1) @{ return (#[1]);@}
  int i;
  def s = #[1];
  for (i=2; i<=size(#); i++)
  @{
    s = s + #[i];
  @}
  return (s);
@}
example
@{ "EXAMPLE:"; echo = 2;
  msum();
  msum(4);
  msum(1,2,3,4);
@}
@c end included file from general.doc:3933
@end smallexample

After typesetting, the library appears in the document as follows (with
one subsection for each procedure):

@c include of docu for template.lib
@include d2t_singular/template_lib.tex

@c ----------------------------------------------------------------------
@node Debugging tools,  , Guidelines for writing a library, General concepts
@section Debugging tools
@cindex Debugging tools

If @sc{Singular} does not come back to the prompt while calling a user
defined procedure, probably a bracket or a @code{"} is missing.  The
easiest way to leave the procedure is to type some brackets or @code{"}
and then @key{RETURN} .

@menu
* Tracing of procedures::
* Source code debugger::
* Break points::
* Printing of data::
* libparse::
@end menu

@c ---------------------------------------
@node Tracing of procedures,Source code debugger,Debugging tools,Debugging tools
@subsection Tracing of procedures
Setting the @code{TRACE} variable to 1 (resp.@: 3) results in a listing of
the called procedures (resp.@: together with line numbers).
If @code{TRACE} is set to 4, @code{Singular}
displays each line before its interpretation and waits for the @key{RETURN}
key being pressed.
@menu
* TRACE var::
@end menu
@iftex
See @ref{TRACE var}.
@end iftex

@strong{Example:}
@smallexample
@c computed example Tracing_of_procedures general.doc:3976 
  proc t1
  @{
    int i=2;
    while (i>0)
    @{ i=i-1; @}
  @}
  TRACE=3;
  t1();
@expansion{} 
@expansion{} entering t1 (level 0)
@expansion{} @{1@}@{2@}@{3@}@{4@}@{5@}@{4@}@{5@}@{6@}@{7@}@{4@}@{5@}@{6@}@{7@}@{4@}@{6@}@{7@}@{8@}
@expansion{} leaving  t1 (level 0)
@c end example Tracing_of_procedures general.doc:3976
@end smallexample

@c ---------------------------------------
@node Source code debugger, Break points, Tracing of procedures, Debugging tools
@subsection Source code debugger
@cindex debugging library code
@cindex source code debugger, sdb
@cindex debugger
@cindex sdb, source code debugger


The source code debugger (sdb) is an experimental feature, it's
interface may change in future versions of Singular.  @*To enable the
use of the source code debugger @sc{Singular} has to be started with the
option @code{-d} or @code{--sdb} (see @ref{Command line options}).


@subsubheading sdb commands
Each sdb command consist of one character which may be followed by
a parameter.
@table @asis
@item @code{b}
print backtrace of calling stack
@item @code{c}
continue
@item @code{e}
edit the current procedure and reload it (current call will be aborted)
@* only available on UNIX systems
@item @code{h},@code{?}
display help screen
@item @code{n}
execute current line, sdb break at next line
@item @code{p} <identifier>
display type and value of the variable given by <identifier>
@item @code{Q}
quit this Singular session
@item @code{q} <flags>
quit debugger, set debugger flags(0,1,2)
@* 0: continue, disable the debugger
@* 1: continue
@* 2: throw an error, return to toplevel
@end table


@subsubheading Syntactical errors in procedures
If @sc{Singular} was started with the command line option @code{-d} or
@code{--sdb} a syntactical error in a procedure will start the
source code debugger instead of returning to the top level with an
error message. The commands @code{q 1} and @code{q 2} are equivalent in this
case.

@subsubheading SDB breakpoints in procedures
@cindex SDB breakpoint
@cindex SDB debugger
Up to seven SDB breakpoints can be set.
To set a breakpoint at a procedure use
@code{breakpoint}. (See @ref{breakpoint}).
@*These breakpoints can be cleared with the command @code{d breakpoint_no}
from within the debugger or with
@code{breakpoint(} proc_name @code{,-1);}.


@c ---------------------------------------
@node Break points, Printing of data, Source code debugger, Debugging tools
@subsection Break points
A break point can be put into a proc by inserting the command @code{~}.
If @code{Singular} reaches a break point it asks for lines of commands
(line-length must be less than 80 characters)
from the user. It returns to normal execution if given an empty line.
See @ref{~}.


@strong{Example:}
@smallexample
proc t
@{
  int i=2;
  ~;
  return(i+1);
@}
t();
@expansion{} -- break point in t --
@expansion{} -- 0: called    from STDIN --
i;               // here local variables of the procedure can be accessed
@expansion{} 2
@expansion{} -- break point in t --

@expansion{} 3
@end smallexample
@sp 2

@c ---------------------------------------
@node Printing of data,libparse,Break points,Debugging tools
@subsection Printing of data
The procedure @code{dbprint} is useful for optional output of data:
it takes 2 arguments and prints the second argument, if the first
argument is positive; it does nothing otherwise.
@menu
* dbprint::
* voice::
@end menu
@ifnothtml
@iftex
See @ref{dbprint}; @ref{voice}.
@end iftex
@end ifnothtml

@c ---------------------------------------
@node libparse,,Printing of data,Debugging tools
@subsection libparse
@cindex libparse

@code{libparse} is a stand-alone program contained in the @sc{Singular}
distribution (at the place where the @sc{Singular} executable program
resides), which cannot be called inside of @sc{Singular}. It is a
debugging tool for libraries which performs exactly the same checks as
@ifset namespaces
the @code{load} command in @sc{Singular}, but generates more output during
@end ifset
@ifclear namespaces
the @code{LIB} command in @sc{Singular}, but generates more output during
@end ifclear
parsing.  @code{libparse} is useful if an error
occurs while loading the library, but the whole block around the line
specified seems to be correct. In these situations the real error might
be hundreds of lines earlier in the library.

@strong{Usage:}
@*@code{libparse [options] singular-library}
@*@strong{Options:}
@table @asis
@item @code{-d} Debuglevel
increases the amount of output during parsing, where Debuglevel is an integer
between 0 and 4. Default is 0.
@item @code{-s}
turns on reporting about violations of unenforced syntax rules
@end table

The following syntax checks are performed in any case:
@itemize @bullet
@item counting of pairs of brackets @{,@} , [,] and (,)
    (number of @{ has to match number of @}, same for [,] and (,) ).
@item counting of "
    ( number of " must be even ).
@item general library syntax
    ( only LIB, static, proc (with parameters, help, body and example)
and comments, i.e // and @code{/* ... */}, are allowed).
@end itemize
Its output lists all procedures that have been parsed successfully:

@smallexample
$ libparse sample.lib
Checking library 'sample.lib'
  Library         function      line,start-eod line,body-eob  line,example-eoe
Version:0.0.0;
g Sample               tab line    9,  149-165   13,  271-298   14,  300-402
l Sample      internal_tab line   24,  450-475   25,  476-496    0,    0-496
@end smallexample

where the following abbreviations are used:
@itemize @bullet
@item g: global procedure (default)
@item l: static procedure, i.e., local to the library.
@end itemize

each of the following is the position of the byte in the library.
@itemize @bullet
@item start: begin of 'proc'
@item eod: end of parameters
@item body: start of procedure body '@{'
@item eob:  end of procedure body '@}'
@item example: position of 'example'
@item eoe: end of example '@}'
@end itemize

Hence in the above example, the first procedure of the library
sample.lib is user-accessible and its name is tab. The procedure
starts in line 9, at character 149. The head of the procedure
ends at character 165, the body starts in line 13 at character 271
and ends at character 298. The example section extends from line 14
character 300 to character 402.

The following example shows the result of a missing close-bracket @} in
line 26 of the library @code{sample.lib}.

@smallexample
LIB "sample.lib";
@expansion{}    ? Library sample.lib: ERROR occurred: in line 26, 497.
@expansion{}    ? missing close bracket '@}' at end of library in line 26.
@expansion{}    ? Cannot load library,... aborting.
@expansion{}    ? error occurred in STDIN line 1: `LIB "sample.lib";`
@end smallexample
